\subsection{Editing terrain} \label{sec:terrain_editing}
Editing terrain is done by changing the values of the scalar field.
When a chunk is first created the values of the scalar field are calculated for each point in the chunk and then saved.
This allows for the scalar field values to be edited.
When a chunk is edited the values of the scalar field are recalculated for the edited points and the points around them.
The player can choose a point to build or mine at.
Values of the scalar field close to that point are then changed based on the pickaxe the player uses and how long they mine for.
The closer the point to the chosen point the more it is affected.
Gaussian distribution is used to calculate exactly how much each point is affected.

Once the values of the scalar field are updated the chunk is regenerated.
This is a very time-consuming process which is why it was moved to a second thread.
The communication between that thread and the main one is described in: \todo{Describe chunk worker}