\section{Architecture problems} \label{sec:architecture_problems}
Clean architecture was very difficult to achieve in this project.
The main reason for this was the fact that it uses three different geometries and each of them is governed by different rules.
For example, as described in \autoref{subsec:practical_considerations}, in hyperbolic geometry the world moves and the camera is stationary.
On the other hand in the case of spherical geometry, it has to be known in which of the two regions described in \autoref{subsec:practical_considerations} the object is currently in.
These are just two examples but there were many more things that only appeared in one of the geometries.
Adding if else statement for each of these would make the code very messy and difficult to read so we decided on a different approach.
We used the strategy pattern to define the behaviors in each of the geometries.
The strategy pattern is a behavioral design pattern that enables the algorithm's behavior to be selected at runtime.
It was originally described in the book \textit{Design Patterns: Elements of Reusable Object-Oriented Software} \cite{Design-Patterns}.
To provide the appropriate strategy to the object we used the factory pattern also described in this book.