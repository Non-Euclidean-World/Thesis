\subsection{Practical considerations} % based on scripts/distortions_calc.py
There are two ways we could implement placing objects in the scene:
\begin{enumerate}
    \item Port the object to non-Euclidean geometry and then use the translation given by \autoref{eq:translation},
    \item Translate the object using ordinary Euclidean translation and then port it to non-Euclidean geometry.
\end{enumerate}
The first option is undesirable, as it may significantly change the relative positions of objects in the scene.
To see why, let's consider two copies of a 2-dimensional rectangle that we will first port to the spherical geometry, and then translate using the non-Euclidean translation.
The rectangle with vertices $a = (-0.5, -0.7), b = (0.5, -0.7), c = (0.5, 0.5), d = (-0.5, 0.5)$ is ported to spherical geometry using \autoref{eq:elliptic-porting}.
As a result, we obtain points on a unit sphere:
\begin{equation*}
    \begin{split}
         & \mathcal{P}(a) = (-0.44057521, -0.61680529,  0.65226123) \\
         & \mathcal{P}(b) =( 0.44057521, -0.61680529,  0.65226123)  \\
         & \mathcal{P}(c) =(0.45936268, 0.45936268, 0.7602446 )     \\
         & \mathcal{P}(d) =(-0.45936268,  0.45936268,  0.7602446 )
    \end{split}
\end{equation*}
If we were to translate the first copy of the rectangle to point $t_1 = (0.3, 0.3)$ and the second copy to $t _2 = (1.3, 1.5)$ in Euclidean geometry, the two copies should meet at the point $(0.8, 0.8)$, see Figure \textcolor{red}{TODO}.
When we perform the translation to point $t_1$ (the corresponding translation target is obtained by porting $t_1$ using \autoref{eq:elliptic-porting}, i.e. the translation target is $q_1 = \mathcal{P}(t_1)$), we get the following vertices:
\begin{equation*}
    \begin{split}
         & T_{q_1}\mathcal{P}(a) =  (-0.22626297, -0.40249305,  0.75499125) \\
         & T_{q_1}\mathcal{P}(b) = ( 0.60970206, -0.44767844,  0.45828748)  \\
         & T_{q_1}\mathcal{P}(c) =(0.61743759, 0.61743759, 0.27773299)      \\
         & T_{q_1}\mathcal{P}(d) =(-0.25706568,  0.66165969,  0.56811146)
    \end{split}
\end{equation*}
The translation to $t_2$ (with $q_2 = \mathcal{P}(t_2)$) gives
\begin{equation*}
    \begin{split}
         & T_{q_2}\mathcal{P}(a) =(0.45082,    0.41172764, 0.23620236)    \\
         & T_{q_2}\mathcal{P}(b) =(1.14026231, 0.19052597, 0.04512474)    \\
         & T_{q_2}\mathcal{P}(c) =( 1.01419188,  1.09955021, -0.20724012) \\
         & T_{q_2}\mathcal{P}(d) =( 0.28308767,  1.31603618, -0.02023601)
    \end{split}
\end{equation*}
Even though we would expect the third vertex of the first copy of the rectangle to be identical to the first vertex of the second copy, there is in fact a difference between the two.
This effect can be seen in Figure \textcolor{red}{TODO}.

The second option isn't unfortunately free of distortions as well.
For example, consider two identical squares of side length $0.5$.
The first one with the bottom-left corner at the point $(0,0)$ and the second one with the corresponding corner at $(0.5, 0.5)$.
After porting to spherical geometry using the \autoref{eq:elliptic-porting}, we get squares with side lengths (listed counter-clockwise starting at the bottom edge):
\begin{equation*}
    \begin{split}
         & 0.5 ,                \\
         & 0.4791055553975933 , \\
         & 0.4791055553975933 , \\
         & 0.5
    \end{split}
\end{equation*}
for the first square and with side lengths
\begin{equation*}
    \begin{split}
         & 0.4804812320876888, \\
         & 0.425392223784045,  \\
         & 0.425392223784045,  \\
         & 0.4804812320876888
    \end{split}
\end{equation*}
for the second square.
The side lengths of the square have been calculated as the lengths of geodesics\footnote{This is the "great-circle distance" equal to $2 \arcsin{(c / 2)}$, where $c$ is the chord length.} between the square's vertices.
As we can see, the side lengths of the ported square are no longer equal to each other, and the distortion increases as the square is farther away from the origin.
This effect can be seen in Figure \textcolor{red}{TODO}.

To minimize the distortions we employed the following method. \textcolor{red}{TODO}.