\subsection{Saving chunks}
Saving chunks is handled by a process similar to the one discussed in \autoref{subsec:loading-and-generating}.
On each frame, the main thread in the \texttt{DeleteChunks} function determines which chunks are too far from the player and thus should be removed from the game and saved to disk.
Positions of these chunks are enqueued into the \texttt{chunksToSave} queue, and their resources are freed.
More precisely, the removal takes place only if the number of chunks in the game exceeds a certain limit.
This is so that chunks are not deleted and loaded again constantly when the player moves back and forth between chunks.

The worker thread dequeues chunks' positions from the \texttt{chunksToSave} queue and saves the scalar field associated with a given chunk on the disk.