\section{Physics}
The Physics component is responsible for collision detection, ray casting, and physical modeling.
Each game object is represented by a \textit{body} (each body has an associated \textit{body handle}) in the physics engine.
This one-to-one mapping is stored in the \texttt{Scene} class in an object of \texttt{SimulationMembers} type.
\texttt{SimulationMembers} allows for adding and removing simulation members as well as accessing handles to bodies in the physics engine associated with a given game object.
Game objects are represented by either simple or composite shapes in the physical simulation.
Simple shapes used in the game include cylinders, capsules, and boxes.
Composite shapes arise by composing simple shapes.
To facilitate debugging, the Physics component also provides a way to extract all shapes from the simulation (and decompose composite shapes) which then can be shown in debug mode.
The body that represents the terrain was created as a separate shape from the terrain mesh triangle-by-triangle.

Game objects can listen and respond to collisions by registering their contact callbacks via the API in \texttt{SimulationManager}.
Such objects implement the \texttt{IContactEventListener} interface whose implementations define the contact callbacks.
The only information about the collision is which two bodies collided and where.

Some game objects can cast rays.
An example of such an object is the player who uses a ray to define a place where terrain modification should take place.
Each ray in the physics engine has an ID number, direction, etc.
An object that wishes to cast rays has to therefore implement an interface (\texttt{IRayCaster}) that exposes all the necessary information.
\todo{Add screenshots of bounding shapes, maybe a screenshot of some huge number of bots to see that we can handle that}.