\subsection{Terrain editing}
The player has three "terrain modifiers", represented with pickaxe symbols, at their disposal.
Using the terrain modifier, the player can edit the game's landscape by building new structures or digging in the ground.
As an example of the "creational capabilities", we show in \autoref{fig:hyper-logo} how the letters making up the game's name could be created inside the game.
\begin{figure}[h]
    \centering
    \includegraphics[width=0.8\textwidth]{chapters/results/sections/gameplay/resources/hyper-logo-night-2.png}
    \caption{The letters of the word "Hyper" created in the game}
    \label{fig:hyper-logo}
\end{figure}

As mentioned before the terrain modifiers can also be used for carving in the terrain.
To illustrate that, in \autoref{fig:tunnel-under-hill} we show a tunnel that was dug through a hill.
\begin{figure}[h]
    \centering
    \includegraphics[width=0.8\textwidth]{chapters/results/sections/gameplay/resources/tunnel-with-car.png}
    \caption{Tunnel dug under a hill}
    \label{fig:tunnel-under-hill}
\end{figure}

% TODO: move this part to discussion/problems
% \todo{I don't know if we should be writing that here or at all.}
% It's important to note that modifying the terrain below a certain level can become "slow".
% This is because the scalar field attains larger values for points with small values of the $y$ coordinate.
% Additionally, the process of terrain modification may appear slightly laggy in spherical space due to bigger chunks' sizes.

