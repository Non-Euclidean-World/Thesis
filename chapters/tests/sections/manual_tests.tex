\section{Manual testing} \label{sec:manual_testing}
Manual testing is responsible for testing the game as a whole.
It tests the results and not particular methods or parts of the code.
Manual tests scenarios were created that describe in detail what should happen in the game after some actions are performed.
This tests cover the entirety of the application and can be viewed at \url{https://github.com/Non-Euclidean-World/HyperTesting}.
The tests are written as Markdown files that have references to image and video files that show the expected behavior of the game.
The test cases are divided into four parts: title, description, prerequisites and steps.
The title describes the name of the test case.
The description describes what the test case is supposed to test.
The prerequisites describe what has to be done before the test case can be performed.
The steps is a table the describes the action that has to be performed and the expected result.
Below is an example test case. 
It is important to note that the test case has been modified to fit the paper format by changing links to the images into references.

\todo{Might want to write a specific test case to put here since all the nice ones have videos in them}

\subsection*{Flashlight works}\label{flashlight-works}

\subsubsection*{Description}\label{description}

Test case for checking if the player can use the flashlight. The test
case should work for all geometries.

\subsubsection*{Prerequisites}\label{prerequisites}

The game is running.

The camera is in 1st person mode.

\subsubsection*{Steps}\label{steps}

\begin{longtable}[]{@{}
  >{\raggedright\arraybackslash}p{(\columnwidth - 4\tabcolsep) * \real{0.3333}}
  >{\raggedright\arraybackslash}p{(\columnwidth - 4\tabcolsep) * \real{0.3333}}
  >{\raggedright\arraybackslash}p{(\columnwidth - 4\tabcolsep) * \real{0.3333}}@{}}
\toprule\noalign{}
\begin{minipage}[b]{\linewidth}\raggedright
Step
\end{minipage} & \begin{minipage}[b]{\linewidth}\raggedright
Action
\end{minipage} & \begin{minipage}[b]{\linewidth}\raggedright
Expected Result
\end{minipage} \\
\midrule\noalign{}
\endhead
\bottomrule\noalign{}
\endlastfoot
1 & Press \texttt{Y} & The flashlight should turn on, see
here. (\autoref{fig:flashlight_on}) \\
2 & Press \texttt{Y} again & The flashlight should turn off \\
\end{longtable}

\begin{figure}[H]
  \centering
  \includegraphics[width=0.5\textwidth]{chapters/tests/resources/flashlight.png}
  \caption{The flashlight is on.}
  \label{fig:flashlight_on}
\end{figure}