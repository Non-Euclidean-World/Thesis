\subsection*{Marching Cubes} \label{subsec:marching_cubes}
Marching cubes is an algorithm for generating a 3D mesh from a scalar field.
It was first described by William E. Lorensen and Harvey E. Cline in 1987 \cite{marching_cubes}.
The main idea behind the algorithm is to divide the scalar field into cubes and then generate a mesh for each one of them.

Some isolevel is chosen and then each point that has a value greater than the isolevel is considered to be "above" the surface and each point with a value lower than the isolevel is considered to be "below" the surface.
We know that a surface has to separate the points that are above it from the points that are below it.
Every cube has 8 vertices and each vertex is either above or below the surface which gives us a total of 256 combinations (some combinations being rotations of others).
For each of these combinations we use a precomputed table to generate a mesh.
The table tells us which edges of the cube are intersected by the surface and how to connect them.
To make the mesh look smoother we interpolate the position of the vertices on the edges based on the values of the scalar field at the vertices.
This is done by using the linear interpolation. % Maybe add formula here?
This gives us a mesh.

However to make the mesh look even smoother we also need to calculate the normal vectors for each vertex.
The normal vectors for each vertex of the scalar field are calculated using \autoref*{eq:normal_vector}
\begin{equation}
    \label{eq:normal_vector}
    n(x, y, z) = \begin{bmatrix}
        s(x + 1, y, z) - s(x - 1, y, z) \\
        s(x, y + 1, z) - s(x, y - 1, z) \\
        s(x, y, z + 1) - s(x, y, z - 1)
      \end{bmatrix}
\end{equation}
where $s$ is the scalar field and $n$ is the normal vector.
These vectors are used to calculate the mesh normals using the same interpolation used for the mesh.

Last part of creating the mesh is assigning the colors to each vertex.
Each vertex of the scalar field is assigned a type which is described in \autoref*{subsec:scalar_field}.
Each type has a color assigned to it.
The color of each vertex of the mesh is calculated by interpolating the colors of the vertices of the scalar field.

