\section{Procedural world generation} \label{sec:implementation_terrain}
\todo{Intro to this section became a mess}
The terrain was designed to be randomly generated so that the player can have a new, unique map every time they play.
The second important part of the design was making sure that the player could edit the terrain in any way they wanted.
As described in \autoref{sec:theory_theory_marching_cubes} the marching cubes algorithm was chosen as the base of the terrain generation process.
This section will describe how we used this algorithm to generate the terrain and allow the player to edit it.

The algorithm consists of the following steps:
\begin{itemize}
    \item Define the scalar field function.
    \item Divide the world into chunks.
    \item Generate the mesh.
\end{itemize}

The chunk generation process is encapsulated inside the \texttt{ChunkFactory} class.
A chunk is generated in two main steps:
\begin{enumerate}
    \item A scalar field of a given size is created.
          The values of the scalar field are generated based on the values of Perlin noise.
          This step is performed by the \texttt{ScalarFieldGenerator} class.
    \item The marching cubes algorithm is used to create a mesh;
          positions and normal vectors of the mesh's vertices are obtained in this step using the \texttt{MeshGenerator} class.
\end{enumerate}
For more information on terrain generation, refer to \autoref{sec:implementation_terrain}.

In what follows, we will provide more details on both terrain generation and modification.

\subsection{Scalar Field} \label{subsec:scalar_field}
The first step in the terrain generation is to generate a scalar field which is a function that assigns a value to each point in 3-dimensional space.
What is important is that this function always returns the same value for the same point.
Another important property is that the function should return similar values for points located close to each other.
Our function returns values for points that have integer coordinates.

Having these properties in mind we decided to use the Perlin noise function.
Perlin noise first introduced by Ken Perlin in 1983 \cite{Perlin-Noise} is often used in computer graphics and in particular in procedural terrain generation.
It is a pseudo-random function that returns values for any point in 3D space.
However, unlike most random functions, it returns similar values for similar points.
This makes it ideal for this game.

The Perlin noise function is used to generate a value for each point in the scalar field.
This value is then modified based on five parameters: number of octaves, initial frequency, frequency multiplier, initial amplitude, and amplitude multiplier.
How these parameters affect the terrain can be seen in \autoref{fig:argument_comparison}.
The values of these parameters depend on the \textit{seed} of the world (there are five sets of options corresponding to five different "terrain styles" or \textit{biomes}).
Each point is also assigned a \textit{type} based on its position that is later used to determine the color of the mesh vertex.

\newpage
\begin{figure}[!htb]
    \centering
    \begin{subfigure}{0.45\textwidth}
        \centering
        \includegraphics[width=0.8\textwidth]{chapters/implementation/sections/terrain/resources/octaves-1.png}
        \caption{Small number of octaves (1).}
    \end{subfigure}
    \hfill
    \begin{subfigure}{0.45\textwidth}
        \centering
        \includegraphics[width=0.8\textwidth]{chapters/implementation/sections/terrain/resources/octaves-5.png}
        \caption{Large number of octaves (5).}
    \end{subfigure}

    \centering
    \begin{subfigure}{0.45\textwidth}
        \centering
        \includegraphics[width=0.8\textwidth]{chapters/implementation/sections/terrain/resources/initial-freq-0.1.png}
        \caption{Small initial frequency (1).}
    \end{subfigure}
    \hfill
    \begin{subfigure}{0.45\textwidth}
        \centering
        \includegraphics[width=0.8\textwidth]{chapters/implementation/sections/terrain/resources/initial-freq-0.5.png}
        \caption{Big initial frequency (0.5).}
    \end{subfigure}

    \centering
    \begin{subfigure}{0.45\textwidth}
        \centering
        \includegraphics[width=0.8\textwidth]{chapters/implementation/sections/terrain/resources/freq-mul-0.5.png}
        \caption{Small frequency multiplier (0.5).}
    \end{subfigure}
    \hfill
    \begin{subfigure}{0.45\textwidth}
        \centering
        \includegraphics[width=0.8\textwidth]{chapters/implementation/sections/terrain/resources/freq-mul-5.png}
        \caption{Big frequency multiplier (5).}
    \end{subfigure}

    \centering
    \begin{subfigure}{0.45\textwidth}
        \centering
        \includegraphics[width=0.8\textwidth]{chapters/implementation/sections/terrain/resources/initial-amp-8.png}
        \caption{Small initial amplitude (8).}
    \end{subfigure}
    \hfill
    \begin{subfigure}{0.45\textwidth}
        \centering
        \includegraphics[width=0.8\textwidth]{chapters/implementation/sections/terrain/resources/initial-amp-32.png}
        \caption{Big initial amplitude (32).}
    \end{subfigure}

    \centering
    \begin{subfigure}{0.45\textwidth}
        \centering
        \includegraphics[width=0.8\textwidth]{chapters/implementation/sections/terrain/resources/amp-mul-0.1.png}
        \caption{Small amplitude multiplier (0.1).}
    \end{subfigure}
    \hfill
    \begin{subfigure}{0.45\textwidth}
        \centering
        \includegraphics[width=0.8\textwidth]{chapters/implementation/sections/terrain/resources/amp-mul-1.png}
        \caption{Big amplitude multiplier (1).}
    \end{subfigure}

    \caption{Scalar field parameter comparison.}
    \label{fig:argument_comparison}
\end{figure}
\subsection{Chunks}
As mentioned before one of the most important things for the terrain was a way to edit it.
Editing the whole terrain at once would be very slow and not very efficient.
Thus the terrain is split into chunks -- cubically shaped, distinct sections of the world.
Each chunk is a separate object and can be edited independently.
This solution is much more efficient but it also causes some problems.

One problem is that the terrain is not continuous.
Every time we edit a chunk we need to make sure that the edges of the chunk are behaving in the same way as the edges of the neighboring chunks.
This is done by making sure that when a function that updates one chunk is called it is also called with the same parameters for other affected chunks.
Without this, the terrain would have holes in it between chunks which is shown in a screenshot from an early version of the game in \autoref{fig:gaps_between_chunks}.

Another problem is that the algorithm we used for generating the terrain, described in \autoref{subsec:marching_cubes}, calculates normal vectors based on the values of the scalar field around the point at which the normal is calculated.
This means that the normal vectors at the edges of the chunks have to be calculated differently.
This is a common problem with the algorithm and it is visualized in \autoref{fig:problem_with_normals_at_chunk_edge}.
The most common solution and the one we used is extending the scalar field by one layer of points around the chunk.
This means that the chunk contains information about the scalar field outside of the chunk itself.
That way the normal vectors can be calculated the same way for all points in the chunk.

\begin{figure}[!htb]
    \centering
    \begin{minipage}{0.45\textwidth}
        \centering
        \includegraphics[width=0.8\textwidth]{chapters/implementation/sections/terrain/resources/chunk_edges_gaps.png}
        \caption{Gaps between chunks.}
        \label{fig:gaps_between_chunks}
    \end{minipage}\hfill
    \begin{minipage}{0.45\textwidth}
        \centering
        \includegraphics[width=0.8\textwidth]{chapters/implementation/sections/terrain/resources/chunk_edges_normals_problem.png}
        \caption{Problem with normals at chunk edges.}
        \label{fig:problem_with_normals_at_chunk_edge}
    \end{minipage}
\end{figure}

There are potentially infinitely many chunks in the world, which is why chunks are only loaded/created when a player is close to them.
They are also unloaded when the player moves far enough away from them.
When that happens they are saved to disk and removed from RAM.
The same thing happens when the game is closed.
\subsection*{Marching Cubes} \label{subsec:marching_cubes}
Marching cubes is an algorithm for generating a 3D mesh from a scalar field.
It was first described by William E. Lorensen and Harvey E. Cline in 1987 \cite{marching_cubes}.
The main idea behind the algorithm is to divide the scalar field into cubes and then generate a mesh for each one of them.

Some isolevel is chosen and then each point that has a value greater than the isolevel is considered to be "above" the surface and each point with a value lower than the isolevel is considered to be "below" the surface.
We know that a surface has to separate the points that are above it from the points that are below it.
Every cube has 8 vertices and each vertex is either above or below the surface which gives us a total of 256 combinations (some combinations being rotations of others).
For each of these combinations we use a precomputed table to generate a mesh.
The table tells us which edges of the cube are intersected by the surface and how to connect them.
To make the mesh look smoother we interpolate the position of the vertices on the edges based on the values of the scalar field at the vertices.
This is done by using the linear interpolation. % Maybe add formula here?
This gives us a mesh.

However to make the mesh look even smoother we also need to calculate the normal vectors for each vertex.
The normal vectors for each vertex of the scalar field are calculated using \autoref*{eq:normal_vector}
\begin{equation}
    \label{eq:normal_vector}
    n(x, y, z) = \begin{bmatrix}
        s(x + 1, y, z) - s(x - 1, y, z) \\
        s(x, y + 1, z) - s(x, y - 1, z) \\
        s(x, y, z + 1) - s(x, y, z - 1)
      \end{bmatrix}
\end{equation}
where $s$ is the scalar field and $n$ is the normal vector.
These vectors are used to calculate the mesh normals using the same interpolation used for the mesh.

Last part of creating the mesh is assigning the colors to each vertex.
Each vertex of the scalar field is assigned a type which is described in \autoref*{subsec:scalar_field}.
Each type has a color assigned to it.
The color of each vertex of the mesh is calculated by interpolating the colors of the vertices of the scalar field.


\subsection{Editing terrain} \label{sec:terrain_editing}
Editing terrain is done by changing the values of the scalar field.
When a chunk is first created the values of the scalar field are calculated for each point in the chunk and then saved.
This allows for the scalar field values to be edited.
When a chunk is edited the values of the scalar field are recalculated for the edited points and the points around them.
The player can choose a point to build or mine at (both of these types of modification are performed using a \textit{pickaxe}).
In the case of mining, the values of the scalar field are increased by a certain amount and in the case of building, they are decreased.
Modification of the scalar field also depends on the pickaxe the player uses and how long they mine for.
Points located closer to the chosen point $c$ are influenced more by the modification whereas the points lying outside of the pickaxe's range are left untouched.
We propose to use a 3D Gaussian function to calculate exactly how much each point is affected to make the terrain look smooth after the modification.
The function is shown in \autoref{eq:gaussian}.

\begin{equation}
    \label{eq:gaussian}
    f(x, y, z) = e^{-a \left((x - c_x)^2 + (y - c_y)^2 + (z - c_z)^2\right)}
\end{equation}

In \autoref{eq:gaussian} $x$, $y$ and $z$ are the coordinates of a point being edited, $c_x$, $c_y$ and $c_z$ are the coordinates of the chosen point (center of the modification) and $a$ is a constant.
The function gives the desired result of gradually decreasing the effect of modification the further a point is from $c$.
The results of the function are added or subtracted from the values of the scalar field depending on the type of modification.

Once the values of the scalar field are updated the chunk's mesh is regenerated.
This is a very time-consuming process which is why it was moved to a second thread.
The communication between that thread and the main one is described in \autoref{sec:chunk-worker}.
