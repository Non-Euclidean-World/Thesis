\subsection{Scalar Field} \label{subsec:scalar_field}
The first step in the terrain generation is to generate a scalar field which is a function that assigns a value to each point in 3-dimensional space.
What is important is that this function always returns the same value for the same point.
Another important property is that the function should return similar values for points located close to each other.
Our function returns values for points that have integer coordinates.

Having these properties in mind we decided to use the Perlin noise function.
Perlin noise first introduced by Ken Perlin in 1983 \cite{Perlin-Noise} is often used in computer graphics and in particular in procedural terrain generation.
It is a pseudo-random function that returns values for any point in 3D space.
However, unlike most random functions, it returns similar values for similar points.
This makes it ideal for this game.

The Perlin noise function is used to generate a value for each point in the scalar field.
This value is then modified based on five parameters: number of octaves, initial frequency, frequency multiplier, initial amplitude, and amplitude multiplier.
How these parameters affect the terrain can be seen in \autoref{fig:argument_comparison}.
The values of these parameters depend on the \textit{seed} of the world (there are five sets of options corresponding to five different "terrain styles" or \textit{biomes}).
Each point is also assigned a \textit{type} based on its position that is later used to determine the color of the mesh vertex.

\newpage
\begin{figure}[!htb]
    \centering
    \begin{subfigure}{0.45\textwidth}
        \centering
        \includegraphics[width=0.8\textwidth]{chapters/implementation/sections/terrain/resources/octaves-1.png}
        \caption{Small number of octaves (1).}
    \end{subfigure}
    \hfill
    \begin{subfigure}{0.45\textwidth}
        \centering
        \includegraphics[width=0.8\textwidth]{chapters/implementation/sections/terrain/resources/octaves-5.png}
        \caption{Large number of octaves (5).}
    \end{subfigure}

    \centering
    \begin{subfigure}{0.45\textwidth}
        \centering
        \includegraphics[width=0.8\textwidth]{chapters/implementation/sections/terrain/resources/initial-freq-0.1.png}
        \caption{Small initial frequency (1).}
    \end{subfigure}
    \hfill
    \begin{subfigure}{0.45\textwidth}
        \centering
        \includegraphics[width=0.8\textwidth]{chapters/implementation/sections/terrain/resources/initial-freq-0.5.png}
        \caption{Big initial frequency (0.5).}
    \end{subfigure}

    \centering
    \begin{subfigure}{0.45\textwidth}
        \centering
        \includegraphics[width=0.8\textwidth]{chapters/implementation/sections/terrain/resources/freq-mul-0.5.png}
        \caption{Small frequency multiplier (0.5).}
    \end{subfigure}
    \hfill
    \begin{subfigure}{0.45\textwidth}
        \centering
        \includegraphics[width=0.8\textwidth]{chapters/implementation/sections/terrain/resources/freq-mul-5.png}
        \caption{Big frequency multiplier (5).}
    \end{subfigure}

    \centering
    \begin{subfigure}{0.45\textwidth}
        \centering
        \includegraphics[width=0.8\textwidth]{chapters/implementation/sections/terrain/resources/initial-amp-8.png}
        \caption{Small initial amplitude (8).}
    \end{subfigure}
    \hfill
    \begin{subfigure}{0.45\textwidth}
        \centering
        \includegraphics[width=0.8\textwidth]{chapters/implementation/sections/terrain/resources/initial-amp-32.png}
        \caption{Big initial amplitude (32).}
    \end{subfigure}

    \centering
    \begin{subfigure}{0.45\textwidth}
        \centering
        \includegraphics[width=0.8\textwidth]{chapters/implementation/sections/terrain/resources/amp-mul-0.1.png}
        \caption{Small amplitude multiplier (0.1).}
    \end{subfigure}
    \hfill
    \begin{subfigure}{0.45\textwidth}
        \centering
        \includegraphics[width=0.8\textwidth]{chapters/implementation/sections/terrain/resources/amp-mul-1.png}
        \caption{Big amplitude multiplier (1).}
    \end{subfigure}

    \caption{Scalar field parameter comparison.}
    \label{fig:argument_comparison}
\end{figure}