\subsection{Editing terrain} \label{sec:terrain_editing}
Editing terrain is done by changing the values of the scalar field.
When a chunk is first created the values of the scalar field are calculated for each point in the chunk and then saved.
This allows for the scalar field values to be edited.
When a chunk is edited the values of the scalar field are recalculated for the edited points and the points around them.
The player can choose a point to build or mine at (both of these types of modification are performed using a \textit{pickaxe}).
In the case of mining, the values of the scalar field are increased by a certain amount and in the case of building, they are decreased.
Modification of the scalar field also depends on the pickaxe the player uses and how long they mine for.
Points located closer to the chosen point $c$ are influenced more by the modification whereas the points lying outside of the pickaxe's range are left untouched.
We propose to use a 3D Gaussian function to calculate exactly how much each point is affected to make the terrain look smooth after the modification.
The function is shown in \autoref{eq:gaussian}.

\begin{equation}
    \label{eq:gaussian}
    f(x, y, z) = e^{-a \left((x - c_x)^2 + (y - c_y)^2 + (z - c_z)^2\right)}
\end{equation}

In \autoref{eq:gaussian} $x$, $y$ and $z$ are the coordinates of a point being edited, $c_x$, $c_y$ and $c_z$ are the coordinates of the chosen point (center of the modification) and $a$ is a constant.
The function gives the desired result of gradually decreasing the effect of modification the further a point is from $c$.
The results of the function are added or subtracted from the values of the scalar field depending on the type of modification.

Once the values of the scalar field are updated the chunk's mesh is regenerated.
This is a very time-consuming process which is why it was moved to a second thread.
The communication between that thread and the main one is described in \autoref{sec:chunk-worker}.