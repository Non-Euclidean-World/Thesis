\subsection{Editing terrain} \label{sec:terrain_editing}
Editing terrain is done by changing the values of the scalar field.
When a chunk is first created the values of the scalar field are calculated for each point in the chunk and then saved.
This allows for the scalar field values to be edited.
When a chunk is edited the values of the scalar field are recalculated for the edited points and the points around them.
The player can choose a point to build or mine at.
Values of the scalar field close to that point are then changed based on the pickaxe the player uses and how long they mine for.
The closer the point to the chosen point $c$ the more it is affected.
We propose to use a 3D Gaussian function to calculate exactly how much each point is affected to make the terrain look smooth after the modification.
The function is shown in \autoref{eq:gaussian}.

\begin{equation}
    \label{eq:gaussian}
    f(x, y, z) = e^{-a \left((x - c_x)^2 + (y - c_y)^2 + (z - c_z)^2\right)}
\end{equation}

In \autoref{eq:gaussian} $x$, $y$ and $z$ are the coordinates of a point being edited, $c_x$, $c_y$ and $c_z$ are the coordinates of the chosen point (center of the modification) and $a$ is a constant.
Moreover, if the distance between the chosen point and the point being edited is greater than the range of the pickaxe, the point is not affected.

Once the values of the scalar field are updated the chunk is regenerated.
This is a very time-consuming process which is why it was moved to a second thread.
The communication between that thread and the main one is described in \autoref{sec:chunk-worker}.