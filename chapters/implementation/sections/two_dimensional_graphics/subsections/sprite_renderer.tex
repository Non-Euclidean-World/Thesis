\subsection{\texttt{SpriteRenderer} class} \label{sprite_renderer}
\todo{this has to be rewritten because it says the same things as textures.tex}
The \texttt{SpriteRenderer} class is responsible for rendering 2D sprites.
It takes a PNG file and a JSON file as input and creates a texture and an array of rectangles that describe the position of each sprite on the texture.
It then uses the texture and the rectangles to render the sprites by passing the rectangle coordinates to the shader.
It only sets the texture once to boost performance.

An exemplary JSON file describing the position of the sprites on the texture is shown below:

\begin{verbatim}
    {
      "width": 10,
      "height": 10,
      "items": [
        {
          "name": "someItem",
          "x": 2,
          "y": 3,
          "width": 2,
          "height": 1
        },
        \dots
      ]
    }
\end{verbatim}


The width and height describe the size of the sprite sheet.
The \texttt{x,y} coordinates describe the position of the sprite on the sprite sheet.
The width and height describe the size of the sprite.
The name of the item is used to identify the sprite.

\todo{Screenshots? FPS, position, ...}