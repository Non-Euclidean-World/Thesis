\section{Game objects management}
A game object, such as a humanoid-looking bot or a car most often can be understood as existing in three different planes simultaneously:
\begin{enumerate}
    \item the visual layer,
    \item the physical layer,
    \item the logical layer.
\end{enumerate}
The visual layer of an object describes how the object looks like during the gameplay.
This side of a game object we will call a \textit{model}.
The information about models (vertices, normal vectors, UV mappings, textures, and animations) is read from COLLADA and PNG files and stored in the \texttt{Model} class.
It's important to note that if multiple game objects look the same, the model information is shared between them in the form of a single \texttt{ModelResource} class instance.
More specifically, the \texttt{ModelResource} class is abstract, and its derived classes are singletons.
Animated models additionally use the \texttt{Aminator} class which provides a way to calculate the bone transforms based on the elapsed time.
Some models are \textit{compound}.
The car model, for example, stores information about the look of the wheels and body separately.

The physical layer defines the physical properties of the game object at any given moment.
Some basic ones include inertia, angular and linear velocity, and friction;
in the case of more complex objects such as a car, the description contains information about \textit{constraints}.
The constraints define the relative positions of different elements of the object and their velocities.
It's worth noting that the position constraints don't have to be "stiff" (they are modeled as springs).
Technically, once a game object is added to the simulation, it is represented by one (or more in the case of compound bodies) \textit{body handle}.
Therefore, each game object has to implement the \texttt{ISimulationMember} interface which has a property that returns the list of all body handles associated with the given game object.

The logical layer is the soul of each object.
NPCs move in a way given by a simple algorithm;
the player can inflict damage to NPCs by shooting projectiles, and so on.
Game objects can react to collisions with other objects by registering for contact callbacks, i.e. implementing the \texttt{IContactEventListener} interface.
They can also move (either on their own, like NPCs, or due to the player input) by registering input callbacks, i.e. implementing the \texttt{IInputSubscriber} interface.

In the remainder of this section, we will cover some of the most important components related to managing game objects.
\subsection{\texttt{Animator} class}\label{subsec:animator-class}

The \texttt{Animator} class is responsible for animating the models loaded by \texttt{ModelLoader}.
It takes animation keyframes and interpolates them using quaternion slerp and vector lerp which are provided by AssimpNet.
It then uses the interpolated keyframes to calculate the transformation matrices for each bone in the model which are later passed to the shader.
The logic is described in more detail in \autoref{sec:theory_theory_models}.
\subsection{\texttt{ModelLoader} class}\label{subsec:model_loader-class}

The model loader is responsible for loading models from the file system.
It loads COLLADA files using AssipmNet to convert them to model class objects.
Each vertex in the mesh of the model must be dependent on at most 4 bones.
It creates VAOs for each mesh of the model, stores them in the Model class object, loads a PNG file, and stores it in a Texture class object.
\todo{I think we could write more about the modeling process itself (not necessarily here).
    Give some screenshots from Blender as well.
    It was a lot of work after all}
\input{chapters/implementation/sections/game_objects_management/subsections/physics_project/physics_project.tex}
\subsection{\texttt{Scene} class} \label{subsec:scene-class}
The \texttt{Scene} class is responsible for storing all the objects in the game the player can interact with.
It is also responsible for releasing the objects' resources.
Some of the objects stored in the \texttt{Scene}'s instance are the following: car, chunks, player, camera, etc.
\subsection{\texttt{*Controller} classes}\label{subsec:controller-classes}
There are nine controller classes in the project:
\texttt{PlayerController},
\texttt{BotsController},
\texttt{ChunksController},
\texttt{ProjectilesController},
\texttt{VehiclesController},
\texttt{LightSourcesController},
\texttt{HudController},
\texttt{BoundingShapesController}, and
\texttt{SkyboxController}.

All the controller classes serve the same purpose.
They are responsible for rendering objects and dealing with object callbacks.
For example, \texttt{ProjectilesController} renders all existing projectiles and also removes dead projectiles from the scene.
Each projectile has an \texttt{IsAlive} property which is read every time a render frame callback is called by the controller to check if the projectile is still alive.

\input{chapters/implementation/sections/game_objects_management/subsections/transporter_classes.tex}
\input{chapters/implementation/sections/game_objects_management/subsections/context_class.tex}