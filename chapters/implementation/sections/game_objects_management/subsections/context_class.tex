\subsection{\texttt{Context} class}
The Context component is responsible for input handling and performing actions on each frame.
There are two sources of user input considered in the video game: keyboard and mouse.
Keys and mouse buttons can be in one of three states: \textit{pressed}, \textit{down}, and \textit{up}.
Additionally, the mouse can also generate events when it's moved.

The \texttt{Context} class stores mappings between different types of input and actions that should be triggered by the given input.
Changes in the input state are tracked by OpenTK and the \texttt{Context} activates relevant actions as a response.
It is a convention that every class that registers new actions in the \texttt{Context} implements the \texttt{IInputSubscriber} interface.
Also, it is worth noting that, for better performance, the Context will only trigger actions associated with the input that has itself been previously registered.