\documentclass[a4paper,11pt,twoside]{report}
% THIS FILE SHOULD BE COMPILED BY pdfLaTeX

% ----------------------   PREAMBLE PART ------------------------------

% ------------------------ ENCODING & LANGUAGES ----------------------

\usepackage[utf8]{inputenc}
%\usepackage[MeX]{polski} % Not needed unless You have a name with polish symbols or sth
\usepackage[T1]{fontenc}
\usepackage[english, polish]{babel}
\usepackage{biblatex}
\usepackage{xcolor}
\usepackage{subcaption}
\addbibresource{references.bib}

\usepackage{amsmath, amsfonts, amsthm, latexsym} % MOSTLY MATHEMATICAL SYMBOLS

\usepackage[final]{pdfpages} % INPUTING TITLE PDF PAGE - GENERATE IT FIRST!
%\usepackage[backend=bibtex, style=verbose-trad2]{biblatex}


\usepackage{commath} % various commands which can make writing math expressions easier --- documentation available at: https://ctan.gust.org.pl/tex-archive/macros/latex/contrib/commath/commath.pdf

\usepackage[hidelinks]{hyperref} % for hyperlinks, for example, urls, references to equations, entries in a bibliography --- hidelinks option removes rectangles around hiperlinks


% ---------------- MARGINS, INDENTATION, LINESPREAD ------------------

\usepackage[inner=20mm, outer=20mm, bindingoffset=10mm, top=25mm, bottom=25mm]{geometry} % MARGINS


\linespread{1.5}
\allowdisplaybreaks         % ALLOWS BREAKING PAGE IN MATH MODE

\usepackage{indentfirst}    % IT MAKES THE FIRST PARAGRAPH INDENTED; NOT NEEDED
\setlength{\parindent}{5mm} % WIDTH OF AN INDENTATION


%---------------- RUNNING HEAD - CHAPTER NAMES, PAGE NUMBERS ETC. -------------------

\usepackage{fancyhdr}
\pagestyle{fancy}
\fancyhf{}
% PAGINATION: LEFT ALIGNMENT ON EVEN PAGES, RIGHT ALIGNMENT ON ODD PAGES 
\fancyfoot[LE,RO]{\thepage} 
% RIGHT HEADER: zawartość \rightmark do lewego, wewnętrznego (marginesu) 
\fancyhead[LO]{\sc \nouppercase{\rightmark}}
% lewa pagina: zawartość \leftmark do prawego, wewnętrznego (marginesu) 
\fancyhead[RE]{\sc \leftmark}

\renewcommand{\chaptermark}[1]{\markboth{\thechapter.\ #1}{}}

% HEAD RULE - IT'S A LINE WHICH SEPARATES HEADER AND FOOTER FROM CONTENT
\renewcommand{\headrulewidth}{0 pt} % 0 MEANS NO RULE, 0.5 MEANS FINE RULE, THE BIGGER VALUE THE THICKER RULE


\fancypagestyle{plain}{
  \fancyhf{}
  \fancyfoot[LE,RO]{\thepage}
  
  \renewcommand{\headrulewidth}{0pt}
  \renewcommand{\footrulewidth}{0.0pt}
}



% --------------------------- CHAPTER HEADERS ---------------------

\usepackage{titlesec}
\titleformat{\chapter}
  {\normalfont\Large \bfseries}
  {\thechapter.}{1ex}{\Large}

\titleformat{\section}
  {\normalfont\large\bfseries}
  {\thesection.}{1ex}{}
\titlespacing{\section}{0pt}{30pt}{20pt} 

    
\titleformat{\subsection}
  {\normalfont \bfseries}
  {\thesubsection.}{1ex}{}


% ----------------------- TABLE OF CONTENTS SETUP ---------------------------

\def\cleardoublepage{\clearpage\if@twoside
\ifodd\c@page\else\hbox{}\thispagestyle{empty}\newpage
\if@twocolumn\hbox{}\newpage\fi\fi\fi}


% THIS MAKES DOTS IN TOC FOR CHAPTERS
\usepackage{etoolbox}
\makeatletter
\patchcmd{\l@chapter}
  {\hfil}
  {\leaders\hbox{\normalfont$\m@th\mkern \@dotsep mu\hbox{.}\mkern \@dotsep mu$}\hfill}
  {}{}
\makeatother

\usepackage{titletoc}
\makeatletter
\titlecontents{chapter}% <section-type>
  [0pt]% <left>
  {}% <above-code>
  {\bfseries \thecontentslabel.\quad}% <numbered-entry-format>
  {\bfseries}% <numberless-entry-format>
  {\bfseries\leaders\hbox{\normalfont$\m@th\mkern \@dotsep mu\hbox{.}\mkern \@dotsep mu$}\hfill\contentspage}% <filler-page-format>

\titlecontents{section}
  [1em]
  {}
  {\thecontentslabel.\quad}
  {}
  {\leaders\hbox{\normalfont$\m@th\mkern \@dotsep mu\hbox{.}\mkern \@dotsep mu$}\hfill\contentspage}

\titlecontents{subsection}
  [2em]
  {}
  {\thecontentslabel.\quad}
  {}
  {\leaders\hbox{\normalfont$\m@th\mkern \@dotsep mu\hbox{.}\mkern \@dotsep mu$}\hfill\contentspage}
\makeatother



% ---------------------- TABLES AD FIGURES NUMBERING ----------------------

\renewcommand*{\thetable}{\arabic{chapter}.\arabic{table}}
\renewcommand*{\thefigure}{\arabic{chapter}.\arabic{figure}}


% ------------- DEFINING ENVIRONMENTS FOR THEOREMS, DEFINITIONS ETC. ---------------

\makeatletter
\newtheoremstyle{definition}
{3ex}%                           % Space above
{3ex}%                           % Space below
{\upshape}%                      % Body font
{}%                              % Indent amount
{\bfseries}%                     % Theorem head font
{.}%                             % Punctuation after theorem head
{.5em}%                          % Space after theorem head, ' ', or \newline
{\thmname{#1}\thmnumber{ #2}\thmnote{ (#3)}}
\makeatother

\theoremstyle{definition}
\newtheorem{theorem}{Theorem}[chapter]
\newtheorem{lemma}[theorem]{Lemma}
\newtheorem{example}[theorem]{Example}
\newtheorem{proposition}[theorem]{Proposition}
\newtheorem{corollary}[theorem]{Corollary}
\newtheorem{definition}[theorem]{Definition}
\newtheorem{remark}[theorem]{Remark}

% --------------------- END OF PREAMBLE PART (MOSTLY) --------------------------





% -------------------------- USER SETTINGS ---------------------------

\usepackage{float}
\usepackage{listings}
\usepackage{xcolor}
\usepackage{menukeys}
\usepackage{array}
\usepackage{graphicx}
\usepackage{caption}
\usepackage{subcaption}

\lstdefinelanguage{json}{
    basicstyle=\normalfont\ttfamily,
    numbers=left,
    numberstyle=\scriptsize,
    stepnumber=1,
    numbersep=8pt,
    showstringspaces=false,
    breaklines=true,
    frame=lines,
    backgroundcolor=\color[rgb]{0.95, 0.95, 0.95},
    stringstyle=\color[rgb]{0.25,0.5,0.35},
    keywordstyle=\color[rgb]{0.5,0.0,0.35},
    commentstyle=\color[rgb]{0.5,0.5,0.5},
    string=[s]{"}{"},
    comment=[l]{//},
    morecomment=[s]{/*}{*/},
    tabsize=2,
    keywords={},
}

\newcommand{\tytul}{Proceduralne generowanie świata i wyzwania związane z renderowaniem w przestrzeniach nieeuklidesowych}
\renewcommand{\title}{Procedural world generation and the challenges of rendering in non-Euclidean spaces}
\newcommand{\type}{Engineer} % Master OR Engineer
\newcommand{\supervisor}{Paweł Kotowski, Ph.D.} % TITLE AND NAME OF THE SUPERVISOR
\newcommand{\todo}[1]{\textbf{\textcolor{red}{TODO: #1}}}


\begin{document}
\sloppy
\selectlanguage{english}

\includepdf[pages=-]{title_page} % THIS INPUTS THE TITLE PAGE

\null\thispagestyle{empty}\newpage

% ------------------ PAGE WITH SIGNATURES --------------------------------

%\thispagestyle{empty}\newpage
%\null
%
%\vfill
%
%\begin{center}
%\begin{tabular}[t]{ccc}
%............................................. & \hspace*{100pt} & .............................................\\
%supervisor's signature & \hspace*{100pt} & author's signature
%\end{tabular}
%\end{center}
%


% ---------------------------- ABSTRACTS -----------------------------

{  \fontsize{12}{14} \selectfont
\begin{abstract}

    \begin{center}
        \title
    \end{center}
    Procedural generation, as a method used in creating video games, has experienced a significant increase in popularity in recent years.
    It has been employed extensively in acclaimed titles such as \textit{Minecraft} and \textit{No Man's Sky} to create virtually infinite worlds that the player is free to interact with.
    On the other hand, a recent game \textit{Hyperbolica} popularized a novel idea of making the virtual worlds even more interesting by setting them in non-Euclidean spaces.
    This work describes the implementation of a video game that incorporates both of the aforementioned concepts.

    The game transforms Euclidean space into a hyperbolic or spherical space, which allows for the use of established algorithms.
    This approach simplifies the implementation of the game logic in different geometries.
    It also allows for changing the curvature of the space at runtime, which is the best way to showcase the differences between the geometries.
    This approach also could theoretically be used to create a game that takes place in a world that is a combination of different geometries.

    The game also features a number of classic game mechanics, such as terrain editing, character animations, day and night cycle, and many more.
    As a result, the game offers a unique experience of exploring and interacting with worlds inherently different from the one we live in.

    \noindent \textbf{Keywords:} Non-Euclidean Geometry, Hyperbolic Geometry, Spherical Geometry, Procedural Generation, Video Games, OpenGL, C\#
    \question{Should keywords start with capital letters?}
\end{abstract}
}

\null\thispagestyle{empty}\newpage


{\selectlanguage{polish} \fontsize{12}{14}\selectfont
\begin{abstract}

    \begin{center}
        \tytul
    \end{center}

    Proceduralne generowanie świata jest techniką zdobywającą rosnącą popularność przy tworzeniu gier komputerowych.
    Zostało ono wykorzystane z powodzeniem m. in. w takich produkcjach jak \textit{Minecraft} czy \textit{No Man's Sky}.
    Z drugiej strony za przyczyną gier takich jak \textit{Hyperbolica}, szersze zainteresowanie zyskała kwestia osadzania gier komputerowych w przestrzeniach nieeuklidesowych.
    Za cel niniejszej pracy obrano stworzenie aplikacji graficznej łączącej obydwa wspomniane elementy.\\

    \todo{Translate from english version when that one is finished.}

    \noindent \textbf{Słowa kluczowe:} slowo1, slowo2, ...
\end{abstract}
}


%% --------------------------- DECLARATIONS ------------------------------------
%
%%
%%	IT IS NECESSARY OT ATTACH FILLED-OUT AUTORSHIP DEECLRATION. SCAN (IN PDF FORMAT) NEEDS TO BE PLACED IN scans FOLDER AND IT SHOULD BE CALLED, FOR EXAMPLE, DECLARATION_OF_AUTORSHIP.PDF. IF THE FILENAME OR FILEPATH IS DIFFERENT, THE FILEPATH IN THE NEXT COMMAND HAS TO BE ADJUSTED ACCORDINGLY.
%%
%%	command attacging the declarations of autorship
%%
%\includepdf[pages=-]{scans/declaration-of-autorship}
%\null\thispagestyle{empty}\newpage
%
%% optional declaration
%%
%%	command attaching the declaataration on granting a license
%%
%\includepdf[pages=-]{scans/declaration-on-granting-a-license}
%%
%%	.tex corresponding to the above PDF files are present in the 3. declarations folder 
%
\null\thispagestyle{empty}\newpage
% ------------------- TABLE OF CONTENTS ---------------------
% \selectlanguage{english} - for English
\pagenumbering{gobble}
\tableofcontents
\thispagestyle{empty}
\newpage % IF YOU HAVE EVEN QUANTITY OD PAGES OF TOC, THEN REMOVE IT OR ADD \null\newpage FOR DOUBLE BLANK PAGE BEFORE INTRODUCTION


% -------------------- THE BODY OF THE THESIS --------------------------------

\null\thispagestyle{empty}\newpage
\pagestyle{fancy}
\pagenumbering{arabic}
\setcounter{page}{11}

\chapter{Introduction}
\markboth{}{Introduction}
\addcontentsline{toc}{chapter}{Introduction}

What is the thesis about? What is the content of it? What is the Author's contribution to it?
\par
WARNING!  In a diploma thesis which is a team project: Description of the work division in the team, including the scope of each co-author’s contribution to the practical part (Team Programming Project) and the descriptive part of the diploma thesis.
\par

Lorem ipsum dolor sit amet, consetetur sadipscing elitr, sed diam nonumyeirmod tempor invidunt ut labore et dolore magna aliquyam erat, sed diamvoluptua. At vero eos et accusam et justo duo dolores et ea rebum. Stet clita kasd gubergren, no sea takimata sanctus est Lorem ipsum dolor sit amet. Lorem ipsum dolor sit amet, consetetur sadipscing elitr, sed diam nonumyeirmod tempor invidunt ut labore et dolore magna aliquyam erat, sed diamvoluptua. At vero eos et accusam et justo duo dolores et ea rebum. Stet clita kasd gubergren, no sea takimata sanctus est Lorem ipsum dolor sit amet.

\todo{What should be included in "Specyfikacja Funkcjonalna" (and where to place it? Should it be included at all?) Just the user stories or maybe the whole deliverable 1?}

\todo{Thesis structure proposal:
  \begin{enumerate}
    \item Introduction
    \item Literature overview / previous work (Hyperbolica, Minecraft, No Man's Sky, papers that we used, etc)
    \item Theoretical foundations (description of the mathematical background for non-Euclidean spaces, description of non-trivial algorithms: marching cubes, (etc.))
    \item Implementation (now it's split between sys. architecture and main components but probably shouldn't, lighting chapter should be moved here)
    \item Tests (unit tests, describe why we test the way we do, cite sources on video game testing, etc)
    \item Results (show the results objectively, "as is")
    \item Discussion and conclusions (Describe what we achieved, what we failed to achieve, discuss what we would do differently, plans for the future, whether we would recommend the technologies/approaches employed in this project to other people that want to do something similar, etc.)
  \end{enumerate}}

\todo{Can the images be located in different places (\texttt{H} vs \texttt{h} modifier)}
\chapter{Theoretical foundations}\label{ch:theoretical_foundations}
\section{Non-Euclidean geometry}

The \textit{Euclidean} geometry is based on a set of five postulates originally given by Euclid.
The postulates read as follows \cite{Weisstein-Postulates}:
\begin{enumerate}
    \item A straight line segment can be drawn joining any two points.
    \item Any straight line segment can be extended indefinitely in a straight line.
    \item Given any straight line segment, a circle can be drawn having the segment as the radius and one endpoint as the center.
    \item All right angles are congruent.
    \item Given any straight line and a point not on it, there exists one and only one straight line that passes through that point and never intersects the first line, no matter how far they are extended. \cite{Weisstein-Parallel}
\end{enumerate}
The fifth postulate, also called the \textit{parallel postulate}, has been for hundreds of years a subject of debate if it can be proven from the former four postulates.
It was discovered, however, that the negation of the parallel postulate doesn't lead to a contradiction\cite{Parallel-Postulate}.
The postulate can be negated in one of two ways.
\begin{itemize}
    \item Given any straight line and a point not on it, there exist \textbf{at least two straight lines} that pass through that point and never intersect the first line, no matter how far they are extended.
    \item Given any straight line and a point not on it, there exists \textbf{no straight line} that passes through that point and never intersects the first line; in other words, there are no parallel lines, since any two lines must intersect.
\end{itemize}

When the parallel postulate is replaced with the first statement, we obtain a new geometry, called the \textit{hyperbolic geometry}.
Similarly, replacing it with the second statement yields the \textit{spherical geometry}.
These geometries are collectively referred to as \textit{non-Euclidean geometries}.

\subsection{Analytic description}
To describe the non-Euclidean geometries analytically, we will follow the approach given by \cite{Szirmay-Kalos2022}.
This approach allows us to view the points of a 3-dimensional non-Euclidean space as a subset of the 4-dimensional \textit{embedding space}.
Since imagining the fourth dimension is not something particularly easy, we will be decreasing the dimensionality whenever we give examples.

The elliptic space can be modeled as a unit 3-sphere, \textit{embedded} in a 4-dimensional Euclidean space.
By saying that a space is embedded in another, we mean that the embedded space inherits the distance from the embedding space.
In this case, the spherical distance, given by $ds^2 = dx^2 + dy^2 + dz^2 + dw^2$ is derived from the Euclidean distance.
This is similar to how we may model the 2-dimensional elliptic space as a sphere, where lines are identified with great circles.
The inner product of two vectors $u$ and $v$ in the Euclidean space is given by
$$ \langle u, v \rangle_E  = u_xv_x + u_yv_y + u_zv_z + u_wv_w.$$
Thus, we can define that a point $p$ belongs to the elliptic geometry if
$$ \langle p, p \rangle_E = 1.$$

The model that we use for the hyperbolic geometry is the so-called \textit{hyperboloid model}.
In this model, points $p$ of the hyperbolic space satisfy the equation
$$p_x^2 + p_y^2 + p_z^2 + p_w^2 = -1,$$
with $p_w > 0$.
The set of these points creates the upper sheet of a hyperboloid, which could be visualized as shown in \autoref{fig:hyperboloid} if the embedding space was Euclidean.
\begin{figure}[h]
    \centering
    \includegraphics[width=0.4\textwidth]{chapters/theoretical_foundations/sections/non-eudlidean-spaces/resources/hyperboloid.png}
    \caption{2-dimensional hyperboloid embedded in Euclidean space}
    \label{fig:hyperboloid}
\end{figure}
However, the hyperbolic space is not embedded in the Euclidean space, but in the \textit{Minkowski space} instead.
In the Minkowski space, the inner product of vectors $u$, $v$ is given by the Lorentzian inner product:
$$\langle u, v \rangle_L = u_xv_x + u_yv_y + u_zv_z - u_wv_w.$$
Thus, the points $p$ belonging to the hyperbolic geometry satisfy the equation
$$\langle p, p \rangle_L = -1.$$
It could be interpreted that they are located on a sphere with a radius of imaginary length $\sqrt{-1}$ (and hence are equidistant from the origin).

To build a unified framework for discussing both types of geometries, we introduce the notion of \textit{sign of curvature}, $\mathcal{L}$, that attains the value $+1$ for spherical, and $-1$ for hyperbolic space.
We also define $\langle u, v \rangle = u_xv_x + u_yv_y + u_zv_z + \mathcal{L}u_wv_w$.

\subsection{Transformations}
We will now define transformations that can be used in non-Euclidean geometries.

\subsubsection{Reflection}
A vector $v_R$ obtained from reflecting vector $v$ on vector $m$, see \autoref{fig:reflection}, can be defined as
\begin{figure}[h]
    \centering
    \includegraphics[width=0.4\textwidth]{chapters/theoretical_foundations/sections/non-eudlidean-spaces/resources/reflection.png}
    \caption{Reflection of $v$ on vector $m$}
    \label{fig:reflection}
\end{figure}
$$v_R = 2 \frac{\langle v, m \rangle}{\lVert m \rVert}\frac{m}{\lVert m \rVert} - v = 2\frac{\langle v, m\rangle}{\langle m,m\rangle}m - v.$$
It can be verified that this definition satisfies the intuitive conditions of a reflection:
\begin{itemize}
    \item The reflected vector $v_R$ lies in the plane spanned by $v$ and $m$,
    \item The transformation is an isometry, i.e. $\lVert u - v \rVert = \lVert u_R - v_R \rVert$.
\end{itemize}
We should also note that given a point $p$ in the geometry, i.e. satisfying $\langle p, p \rangle = \mathcal{L}$, its reflection, $p'$, is also in the geometry.

\subsubsection{Translation}
Just like in Euclidean space, we can define translation in terms of an even number of reflections.
More specifically, the translation will be defined by specifying two points: \textit{geometry origin}, $g = (0, 0, 0, 1)$ and \textit{translation target}, $q$, which is the point that the geometry origin is translated to.
Now we can define that the translation is the composition of two reflections: one on the vector $m_1 = g$ and the second one on the vector $m_2 = g + q$, which is halfway between $g$ and $q$.
Applying the first reflection to an arbitrary point $p$ gives a point
$$p' = 2 \frac{\langle p, g \rangle}{\langle g, g \rangle}g - p = 2p_w g - p,$$
and the second reflection applied to $p'$ yields a point
\begin{equation} \label{eq:translation}
    p'' = 2 \frac{\langle p', g + q \rangle}{\langle g + q, g + q \rangle}(g + q) - p'
    = 2 p_w q + p - \frac{p_w + \mathcal{L}\langle p, q \rangle}{1 + q_w}(g + q).
\end{equation}
The effect of applying translation to an arbitrary point $a$ is shown in \autoref{fig:translation}.\\
\begin{figure}[h]
    \centering
    \includegraphics[width=0.4\textwidth]{chapters/theoretical_foundations/sections/non-eudlidean-spaces/resources/translation.png}
    \caption{Translation of a point $a$}
    \label{fig:translation}
\end{figure}
It can be verified that the geometry origin $g$ is indeed translated to point $g'' = q$.
We can evaluate the formula for the basis vectors $i = (1, 0, 0, 0)$, $j = (0, 1, 0, 0)$, $k = (0, 0, 1, 0)$, and $l = (0, 0, 0, 1)$ obtaining the translation matrix
$$
    T(q) = \begin{bmatrix}
        1 - \mathcal{L}\frac{q_x^2}{1 + q_w} & -\mathcal{L}\frac{q_x q_y}{1 + q_w}  & -\mathcal{L}\frac{q_x q_z}{1 + q_w}  & -\mathcal{L} q_x \\
        -\mathcal{L}\frac{q_y q_x}{1 + q_w}  & 1 - \mathcal{L}\frac{q_y^2}{1 + q_w} & -\mathcal{L}\frac{q_y q_z}{1 + q_w}  & -\mathcal{L} q_y \\
        -\mathcal{L}\frac{q_z q_x}{1 + q_w}  & -\mathcal{L}\frac{q_z q_y}{1 + q_w}  & 1 - \mathcal{L}\frac{q_z^2}{1 + q_w} & -\mathcal{L} q_z \\
        q_x                                  & q_y                                  & q_z                                  & q_w
    \end{bmatrix}
$$
It can be seen that the translation is an isometry since the row vectors of the matrix are orthonormal.

\subsubsection{Rotation}
It can be shown that a rotation about an axis through the origin is the same as the Euclidean rotation about the same axis \cite{Philips-Mark-Gunn1992}.

\subsubsection{Camera transformation}
% The camera transformation is defined in terms of the \textit{camera position} $e$, \textit{gaze direction} $g$, and the \textit{view-up vector} $t$.
% The camera position is a location that the camera "sees from", the gaze direction is a vector in the direction the camera is looking, and the view-up vector is a vector that points "to the sky".
% Using these vectors we can define the \textit{view space}, i.e. the camera's local coordinate system with the camera at the geometry origin and basis vectors $i'$, $j'$, and $k'$ defined as follows:
% \begin{equation*}
%     \begin{split}
%         k' = -\frac{g}{\lVert g \rVert}                    \\
%         i' = \frac{t \times k'}{\lVert t \times k' \rVert} \\
%         j' = k' \times i'
%     \end{split}
% \end{equation*}

The camera transformation allows us to describe the scene from the viewer's perspective.
The transformation is defined in terms of the \textit{eye position} $e$, and three orthonormal vectors in the tangent space of the eye:
\begin{enumerate}
    \item the right direction $i'$,
    \item the up direction $j'$, and
    \item the negative view direction $k'$.
\end{enumerate}
An example in \autoref{fig:tangent-space} shows the tangent space of the eye, with the $e$ vector marked green, $-k'$ marked blue, and $i'$ marked orange.\\
\begin{figure}[h]
    \centering
    \includegraphics[width=0.4\textwidth]{chapters/theoretical_foundations/sections/non-eudlidean-spaces/resources/tangent-space.png}
    \caption{Tangent space of the camera}
    \label{fig:tangent-space}
\end{figure}
The transformation can be described by the matrix
\begin{equation*}
    V =
    \begin{bmatrix}
        i'_x            & j'_x            & k'_x            & \mathcal{L}e_x \\
        i'_y            & j'_y            & k'_y            & \mathcal{L}e_y \\
        i'_z            & j'_z            & k'_z            & \mathcal{L}e_z \\
        \mathcal{L}i'_w & \mathcal{L}j'_w & \mathcal{L}l'_w & e_w
    \end{bmatrix}
\end{equation*}
As the result of the transformation, the eye position is mapped to the geometry origin $g$.
Furthermore, the vectors $i'$, $j'$, and $k'$ are mapped to $i$, $j$, and $k$, respectively.

\subsubsection{Perspective transformation}
The perspective transformation is described using a projection matrix $P$.
The projection matrix we use in spherical geometry is identical to the one used in the \textit{Unity} implementation of \cite{Szirmay-Kalos2022} (see \url{https://github.com/mmagdics/noneuclideanunity}).
It is parameterized by the \textit{near plane distance} $n$, \textit{far plane distance} $f$, \textit{aspect ratio} ASP, and \textit{field of view} FOV:
\begin{equation*}
    P =
    \begin{bmatrix}
        s_x & 0   & 0  & 0  \\
        0   & s_y & 0  & 0  \\
        0   & 0   & 0  & -1 \\
        0   & 0   & -n & 0
    \end{bmatrix},
\end{equation*}
where $s_x = 2n / (r - l)$, $s_y = 2n / (t - b)$, and $r$, $l$, $t$, $b$ are defined in terms of $u = f \tan(\mathrm{FOV})$:
\begin{equation*}
    r = u \cdot \mathrm{ASP}, \,
    l = -u \cdot \mathrm{ASP}, \,
    t = u, \,
    b =  -u.
\end{equation*}
For hyperbolic and Euclidean geometries, the standard projection matrix is used.

\subsubsection*{Porting objects}
The positions of objects in the scene are specified in a 3-dimensional Euclidean space.
They are then "transported" or \textit{ported} to a non-Euclidean space of choice.
One possible mapping that could be used for this purpose is called the exponential map.
For a given point $p$ in the 3-dimensional Euclidean space with coordinates $(x, y, z)$ the mapping to elliptic geometry is given by
\begin{equation} \label{eq:elliptic-porting}
    \mathcal{P}_E(p) = (p / d \sin(d), \cos(d)),
\end{equation}
and for hyperbolic space, it is given by
\begin{equation*}
    \mathcal{P}_E(p) = (p / d \sinh(d), \cosh(d)).
\end{equation*}
The effect of porting a 1-dimensional point $p$ onto a 1-dimensional elliptic space can be seen in \autoref{fig:exp-map}.
\begin{figure}[h]
    \centering
    \includegraphics[width=0.4\textwidth]{chapters/theoretical_foundations/sections/non-eudlidean-spaces/resources/exp-map.png}
    \caption{Exponential map}
    \label{fig:exp-map}
\end{figure}
\subsection{Practical considerations} % based on scripts/distortions_calc.py
There are two ways we could implement placing objects in the scene:
\begin{enumerate}
    \item Port the object to non-Euclidean geometry and then use the translation given by \autoref{eq:translation},
    \item Translate the object using ordinary Euclidean translation and then port it to non-Euclidean geometry.
\end{enumerate}
The first option is undesirable, as it may significantly change the relative positions of objects in the scene.
To see why, let's consider two copies of a 2-dimensional rectangle that we will first port to the spherical geometry, and then translate using the non-Euclidean translation.
The rectangle with vertices $a = (-0.5, -0.7), b = (0.5, -0.7), c = (0.5, 0.5), d = (-0.5, 0.5)$ is ported to spherical geometry using \autoref{eq:elliptic-porting}.
As a result, we obtain points on a unit sphere:
\begin{equation*}
    \begin{split}
         & \mathcal{P}(a) = (-0.44057521, -0.61680529,  0.65226123) \\
         & \mathcal{P}(b) =( 0.44057521, -0.61680529,  0.65226123)  \\
         & \mathcal{P}(c) =(0.45936268, 0.45936268, 0.7602446 )     \\
         & \mathcal{P}(d) =(-0.45936268,  0.45936268,  0.7602446 )
    \end{split}
\end{equation*}
If we were to translate the first copy of the rectangle to point $t_1 = (0.3, 0.3)$ and the second copy to $t _2 = (1.3, 1.5)$ in Euclidean geometry, the two copies should meet at the point $(0.8, 0.8)$, see Figure \textcolor{red}{TODO}.
When we perform the translation to point $t_1$ (the corresponding translation target is obtained by porting $t_1$ using \autoref{eq:elliptic-porting}, i.e. the translation target is $q_1 = \mathcal{P}(t_1)$), we get the following vertices:
\begin{equation*}
    \begin{split}
         & T_{q_1}\mathcal{P}(a) =  (-0.22626297, -0.40249305,  0.75499125) \\
         & T_{q_1}\mathcal{P}(b) = ( 0.60970206, -0.44767844,  0.45828748)  \\
         & T_{q_1}\mathcal{P}(c) =(0.61743759, 0.61743759, 0.27773299)      \\
         & T_{q_1}\mathcal{P}(d) =(-0.25706568,  0.66165969,  0.56811146)
    \end{split}
\end{equation*}
The translation to $t_2$ (with $q_2 = \mathcal{P}(t_2)$) gives
\begin{equation*}
    \begin{split}
         & T_{q_2}\mathcal{P}(a) =(0.45082,    0.41172764, 0.23620236)    \\
         & T_{q_2}\mathcal{P}(b) =(1.14026231, 0.19052597, 0.04512474)    \\
         & T_{q_2}\mathcal{P}(c) =( 1.01419188,  1.09955021, -0.20724012) \\
         & T_{q_2}\mathcal{P}(d) =( 0.28308767,  1.31603618, -0.02023601)
    \end{split}
\end{equation*}
Even though we would expect the third vertex of the first copy of the rectangle to be identical to the first vertex of the second copy, there is in fact a difference between the two.
This effect can be seen in Figure \textcolor{red}{TODO}.

The second option isn't unfortunately free of distortions as well.
For example, consider two identical squares of side length $0.5$.
The first one with the bottom-left corner at the point $(0,0)$ and the second one with the corresponding corner at $(0.5, 0.5)$.
After porting to spherical geometry using the \autoref{eq:elliptic-porting}, we get squares with side lengths (listed counter-clockwise starting at the bottom edge):
\begin{equation*}
    \begin{split}
         & 0.5 ,                \\
         & 0.4791055553975933 , \\
         & 0.4791055553975933 , \\
         & 0.5
    \end{split}
\end{equation*}
for the first square and with side lengths
\begin{equation*}
    \begin{split}
         & 0.4804812320876888, \\
         & 0.425392223784045,  \\
         & 0.425392223784045,  \\
         & 0.4804812320876888
    \end{split}
\end{equation*}
for the second square.
The side lengths of the square have been calculated as the lengths of geodesics\footnote{This is the "great-circle distance" equal to $2 \arcsin{(c / 2)}$, where $c$ is the chord length.} between the square's vertices.
As we can see, the side lengths of the ported square are no longer equal to each other, and the distortion increases as the square is farther away from the origin.
This effect can be seen in Figure \textcolor{red}{TODO}.

To minimize the distortions we employed the following method. \textcolor{red}{TODO}.
\section{Marching Cubes} \label{sec:theory_theory_marching_cubes}
One of the requirements for our project was to incorporate terrain generation.
Numerous algorithms exist for this purpose, and the chosen algorithm for our project is known as marching cubes.
This algorithm was selected due to its simplicity, versatility, and standardization.
Marching cubes is a straightforward algorithm that can be easily modified to suit different requirements, as explained in detail in \autoref{sec:system_architecture_terrain_generation}.
Furthermore, it is widely used in various applications and is well-documented, making its understanding and implementation easier.

The concept of marching cubes was first introduced by William E. Lorensen and Harvey E. Cline in 1987 \cite{Marching-Cubes}.
The fundamental idea behind this algorithm is to generate a mesh from a scalar field.
An isolevel is chosen, 0 being the most common choice and the one we used.
Points with values greater than the isolevel are considered to be "above" the surface, while points with values lower than the isolevel are considered to be "below" the surface.
The world is divided into cubes, and for each cube, the algorithm determines which of its vertices are above and below the surface.
Based on this information, a mesh is created to separate the vertices above the surface from those below it.
An example of this process is illustrated in \autoref{fig:cube_example}, where $v0$ is below the surface, while the remaining vertices are above it.
It is important to note that the same effect would be achieved if $v0$ were above the surface and the other vertices were below it.

\begin{figure}[H]
    \centering
    \includegraphics[width=0.8\textwidth]{chapters/theoretical_foundations/sections/marching_cubes/resources/cube-example.png}
    \caption["Marching" cubes with only $v0$ below the surface.]{"Marching" cubes with only $v0$ below the surface. Source: \url{https://polycoding.net/marching-cubes/}}
    \label{fig:cube_example}
\end{figure}

Each cube consists of 8 vertices, and each vertex is classified as either above or below the surface, resulting in a total of 256 possible combinations.
However, there are only 15 unique combinations, all of them depicted in \autoref{fig:marching_cubes_configurations}, with the remaining combinations being rotations and reflections of these 15 cases.
For each of these unique combinations, a precomputed table is utilized to generate the corresponding mesh.
This table provides information about which edges of the cube are intersected by the surface and how to connect them.

\begin{figure}[H]
    \centering
    \includegraphics[width=0.8\textwidth]{chapters/theoretical_foundations/sections/marching_cubes/resources/marching-cubes-configurations.png}
    \caption[Unique marching cubes configurations]{Unique marching cubes configurations. Source: \url{https://polycoding.net/marching-cubes/}}
    \label{fig:marching_cubes_configurations}
\end{figure}

\section{Models} \label{sec:theory_theory_models}
The game includes two characters and a car.
Both characters use the same model, and the same texture with different colors.
The car is simpler than the characters as it does hot have any animations.
Because of that this section will only focus on the player model.

The model can be divided into two parts.
The mesh part and the animation part.
What the two parts have in common is that they are both created in Blender, exported to a COLLADA file and loaded into the game using AssimpNet.
Both of these parts are described in this section.


\subsection{Model Mesh} \label{sec:theory_theory_models_mesh}
The model mesh contains the vertices, normals, texture coordinates and indices of the model.
The vertices are the points that make up the model and the normals are automatically assigned by Blender.
The texture coordinates are the coordinates of the texture that is mapped onto the model.
The texture however is a two dimensional image and the model is three dimensional so the mapping is not trivial.
The model mesh has seams that define how a mesh is unwrapped.
These are selected edges that are marked as seams in Blender.
Cutting along these seams will result in a set of faces that can be laid out flat.
This is called the UV map.
This is how the texture is mapped onto the model.
A UV map is shown in \autoref{fig:uv_map}.
For example, the helmet is cut into two parts (the front and the back) and laid out flat in the top right corner of the texture.
This texture describes the color of each pixel of the model.
The whole model can be seen in \autoref{fig:player_model}.

\begin{figure}[h]
    \centering
    \begin{subfigure}{0.45\textwidth}
        \centering
        \includegraphics[width=0.8\textwidth]{chapters/theoretical_foundations/sections/models/resources/Texture.png}
        \caption{UV map}
        \label{fig:uv_map}
    \end{subfigure}
    \hfill
    \begin{subfigure}{0.45\textwidth}
        \centering
        \includegraphics[width=0.8\textwidth]{chapters/theoretical_foundations/sections/models/resources/Model.png}
        \caption{Player model}
        \label{fig:player_model}
    \end{subfigure}

    \caption{Player model and the texture.}
\end{figure}

\subsection{Model Animation} \label{sec:theory_theory_models_animation}
Model animations make use of a \textit{model armature} (also called \textit{skeleton}) which can be seen in \autoref{fig:armature}.
The armature is a set of bones that have different relations to each other and the mesh.
The bones create a tree structure with the root being the torso bone.
Most other bones are children of the torso bone or children of children of the torso bone.
This relation allows the bones to inherit transformations from their parents.
For example, if a leg bone moves the foot bone will follow.

It is worth noting that bone structure does not have to resemble the human bone structure.
For example in \autoref{fig:armature_foot} we can see a set of bones that are outside of the body.
These bones are not used to deform the mesh but to interact with other bones.
Among other things, these guarantee that the knee does not bend in the wrong direction.


\begin{figure}[h]
    \centering
    \begin{subfigure}{0.45\textwidth}
        \centering
        \includegraphics[width=0.8\textwidth]{chapters/theoretical_foundations/sections/models/resources/Armature.png}
        \caption{Armature.}
        \label{fig:armature}
    \end{subfigure}
    \hfill
    \begin{subfigure}{0.45\textwidth}
        \centering
        \includegraphics[width=0.8\textwidth]{chapters/theoretical_foundations/sections/models/resources/ArmatureFoot.png}
        \caption{Abnormal bones.}
        \label{fig:armature_foot}
    \end{subfigure}

    \caption{Armature.}
\end{figure}

The bones need to be connected to the mesh.
Each vertex is assigned a set of bones that influence it with different weights.
When bones move the vertices move with them according to how much they move and how much a certain bone influences a certain vertex.
Assigning weights to vertices in Blender is shown in \autoref{fig:vertex_weights}.
As can be seen in the figure, it is done by selecting a bone and then painting the vertices with a certain weight.

\begin{figure}[h]
    \centering
    \includegraphics[width=0.45\textwidth]{chapters/theoretical_foundations/sections/models/resources/WeightPaint.png}
    \caption{Vertex weights.}
    \label{fig:vertex_weights}
\end{figure}

To define an animation a set of \textit{keyframes} is used.
The process of creating the keyframes can be seen in \autoref{fig:keyframes}.
A keyframe is a set of transformations for each bone at a certain time.
Once the keyframes are defined the animation is interpolated between them to produce a smooth animation.
In the game, there are two animations: walking and running, both of which are looped to create a continuous animation when the player or an NPC moves.

\begin{figure}[h]
    \centering
    \includegraphics[width=1\textwidth]{chapters/theoretical_foundations/sections/models/resources/DopeSheet.png}
    \caption{Keyframes.}
    \label{fig:keyframes}
\end{figure}


The whole process of rendering an animation is as follows:
\begin{enumerate}
    \item The model along with the armature and keyframes are loaded into the game.
    \item The timer is started.
    \item The bone transformations for the current frame are calculated based on the keyframes and the timer, by interpolating between the keyframes.
    \item The bone transformations are passed into the shader.
    \item For each vertex the shader iterates over the bones that influence it and calculates the final position of the vertex based on the bone transformations and their weights.
    \item The animation is rendered.
\end{enumerate}
\section{Day night cycle}
Even though the game can be played in non-Euclidean spaces which makes it inherently unrealistic, we decided to add some elements that would make the scenes portrayed in the game feel familiar.
In particular, one property of the real world that we wanted to capture in the game was the daytime cycle.
In the game, the full cycle is 10 minutes long, with 5 minutes long daytime and 5 minutes long nighttime.
We also added transitions between day and night to give the Earth-like experience of sunrise and sunset.
The scene during various times of the day is shown in \autoref{fig:cycle}.

\begin{figure*}[h]
    \centering
    \begin{subfigure}[b]{0.475\textwidth}
        \centering
        \includegraphics[width=\textwidth]{chapters/theoretical_foundations/sections/day_night_cycle/resources/day.png}
        \caption[]%
        {{\small Day}}
        \label{fig:cycle-day}
    \end{subfigure}
    \hfill
    \begin{subfigure}[b]{0.475\textwidth}
        \centering
        \includegraphics[width=\textwidth]{chapters/theoretical_foundations/sections/day_night_cycle/resources/night.png}
        \caption[]%
        {{\small Night}}
        \label{fig:cycle-night}
    \end{subfigure}
    \vskip\baselineskip
    \begin{subfigure}[b]{0.475\textwidth}
        \centering
        \includegraphics[width=\textwidth]{chapters/theoretical_foundations/sections/day_night_cycle/resources/sunrise.png}
        \caption[]%
        {{\small Sunrise}}
        \label{fig:cycle-sunrise}
    \end{subfigure}
    \hfill
    \begin{subfigure}[b]{0.475\textwidth}
        \centering
        \includegraphics[width=\textwidth]{chapters/theoretical_foundations/sections/day_night_cycle/resources/sunset.png}
        \caption[]%
        {{\small Sunset}}
        \label{fig:cycle-sunset}
    \end{subfigure}
    \caption[]
    {\small Day night cycle in the game}
    \label{fig:cycle}
\end{figure*}

The implementation of the day-night cycle relies on two components: directional lighting (described in \autoref{subsec:directional-lighting}) which corresponds to the light coming from the sun and a \textit{skybox} representing the sky.
Conceptually, a skybox is a cube made out of 6 images, one per each side, that encompasses the scene thus creating an illusion that the world is much bigger than it is in reality.
A skybox can be implemented in OpenGL using a special type of texture, a \textit{cubemap}, i.e. a texture that contains 6 individual 2D textures.
In the game, we're using images of the night sky obtained from an HDR file \url{https://www.reddit.com/r/blender/comments/3ebzwz/free_space_hdrs_1/} using an online utility program \url{https://matheowis.github.io/HDRI-to-CubeMap/}.
The images were then slightly edited by applying Gaussian blur to make the stars appear larger.
The vertices of the cube passed to the vertex shader are transformed using the model, view, and projection matrices.
The model matrix is responsible for rotating the skybox (similarily to how stars appear to move across the night sky as the Earth is rotating).
The view matrix has to be modified so that the skybox doesn't move along with the camera.
The part responsible for translation can be removed from the view matrix by replacing the last row the the view matrix with the vector $ \begin{bmatrix} 0 & 0 & 0 & 1 \end{bmatrix} $ \cite{LearnOpenGL-Cubemaps}.

The day is split into \textit{phases}, each characterized by the color of the sky at the zenith, the sky's color at the horizon, the color of the sun, and \textit{stars' visibility factor}.
In the fragment shader, we determine the color of each fragment.
This is done by first obtaining the zenith and horizon colors by interpolating between the corresponding colors for the previous and the next phase based on the current time.
In the same manner, we obtain the current stars' visibility factor.
Then, we obtain a sky color for a given fragment by interpolating between the zenith and horizon colors based on the height of the fragment\footnote{The "position" of a fragment in this context is given by world space coordinates, normalized so that we're treating the points as located on a sphere (\textit{skydome}). The height is then simply the $y$ coordinate of the fragment's position.}.
The sun's position is given by a vector $s$ that rotates at the same rate as the skybox.
Calculating a dot product $d$ of $s$ with the current fragment's position allows us to easily draw the sun and the sun glare by mixing the sky's color with the sun's color in proportions depending on $d$.
As the last step, we mix the pure-day-time color of the sky with the pure-night-time texture of the stars in proportions given by the stars' visibility factor.

The day-night cycle hasn't been implemented for the spherical space, as the terrain in the spherical space fully "encloses" the scene leaving no way of seeing anything "outside".
\section{Lighting}\label{sec:theory_theory_lighting}
Lighting is an important aspect of the game, making it more immersive and has a major impact on how it is perceived by the player.
Artificial light sources present in the game are also indispensable when exploring the world during the in-game night.
In the game, we use three types of light casters: \textit{directional lights}, \textit{point lights}, and \textit{spotlights}.
As the lighting model, we used the \textit{Phong lighting model}.
In this model, light is considered to have 3 components:
\begin{itemize}
    \item ambient lighting $I_a$ (with coefficient $k_a$),
    \item diffuse lighting $I_d$ (with coefficient $k_d$),
    \item specular lighting $I_s$ (with coefficient $k_s$ and material shiness constant $\alpha$).
\end{itemize}
In the case of $N$ light sources in the scene, the total illumination is calculated using the formula
\begin{equation}
    L = k_a I_a + \sum_{i = 1}^N {k_d I_{i, d} \max(0, \langle n, l_i, \rangle) + k_s I_{i, s} \max(0, \langle r_i, v \rangle^\alpha )},
\end{equation}
where $n$ is the normal vector of the fragment, $l$ is the vector pointing from the fragment to the light source, $r$ is the reflection of $l$ on $n$, and $v$ is the vector pointing towards the viewer (all of the aforementioned vectors are assumed to be normalized).
It's important to note that $\langle \cdot, \cdot\rangle$ is the inner product as defined by \autoref{eq:gen-inner-prod}.
The reflected light vector $r$ is calculated using the usual formula
\begin{equation*}
    r = 2 \langle l, n \rangle n - l.
\end{equation*}

\subsection{Directional lighting} \label{subsec:directional-lighting}
During the in-game daytime, the directional light is used to represent the light cast by the sun.
The light direction $l$ is the vector pointing toward the sun.
During the night the directional light is much dimmer but still present.
In this case, the light direction $l$ is pointing toward an imaginary light source ("the stars") which is rotating along with the sky.
\autoref{fig:directional-light} shows the terrain and other game objects illuminated by the directional light of orange color.
\begin{figure}[h]
    \centering
    \includegraphics[width=0.8\textwidth]{chapters/theoretical_foundations/sections/lighting/resources/directional-light.png}
    \caption{Directional light}
    \label{fig:directional-light}
\end{figure}
\subsection{Point lights}
In the game, spotlights are represented by white spherical lamps.

In Euclidean geometry, assuming that a point light is placed at a point $p$ and that the current fragment's position is given by $f$, we can calculate the light direction vector $l$ simply as
\begin{equation*}
    l = \frac{p - f}{d_E(p, f)},
\end{equation*}
where $d_E$ is the usual Euclidean distance, i.e.
\begin{equation} \label{eq:dist-euclidean}
    d_E(a, b) = \sqrt{\langle a - b, a - b\rangle_E}.
\end{equation}
For non-Euclidean geometries, we use modified formulas given by \cite{Szirmay-Kalos2022}, namely
\begin{equation}
    l = \frac{p - f \cos(d_S(p, f))}{\sin(d_S(p, f))}
\end{equation}
for spherical geometry and
\begin{equation}
    l = \frac{p - f\cosh(d_H(p, f))}{\sinh(d_H(p, f))}
\end{equation}
for hyperbolic geometry.
The spherical and hyperbolic distances $d_S$ and $d_H$ are given by
\begin{equation} \label{eq:dist-spherical}
    d_S(a, b) = \cos^{-1}(|\langle a, b \rangle_E|)
\end{equation}
and
\begin{equation} \label{eq:dist-hyperbolic}
    d_H(a, b) = \cosh^{-1}(-\langle a, b \rangle_L)
\end{equation}
respectively.

The important difference between a point light and a directional light is the \textit{attenuation factor} $a$.
Attenuation represents how the light's strength diminishes over distance.
It can be expressed as a reciprocal of a quadratic function:
\begin{equation}
    a = \frac{1}{K_c + K_l d + K_q d^2},
\end{equation}
where $d$ is the distance of the fragment from the source that can be calculated using one of the formulas \ref{eq:dist-euclidean}, \ref{eq:dist-spherical}, or \ref{eq:dist-hyperbolic} depending on the geometry and $K_c$, $K_l$, and $K_q$ are constants specific to the light source's strength.
Multiplying the light by the attenuation factor gives the desired effect of a realistic point light source such as a lamp, see \autoref{fig:point-lights}.
\begin{figure}[!htb]
    \centering
    \includegraphics[width=0.8\textwidth]{chapters/theoretical_foundations/sections/lighting/resources/point-lights.png}
    \caption{Point lights}
    \label{fig:point-lights}
\end{figure}
Point lights in hyperbolic and spherical geometry are shown in \autoref{fig:point-light-non-euc}.
\begin{figure*}[!htb]
    \centering
    \begin{subfigure}[b]{0.475\textwidth}
        \centering
        \includegraphics[width=\textwidth]{chapters/theoretical_foundations/sections/lighting/resources/point-light-hyperbolic.png}
        \caption[]%
        {{\small Hyperbolic space}}
        \label{fig:point-light-hyperbolic}
    \end{subfigure}
    \hfill
    \begin{subfigure}[b]{0.475\textwidth}
        \centering
        \includegraphics[width=\textwidth]{chapters/theoretical_foundations/sections/lighting/resources/point-light-spherical.png}
        \caption[]%
        {{\small Spherical space}}
        \label{fig:point-light-spherical}
    \end{subfigure}
    \caption[]
    {\small Point lights in non-Euclidean spaces}
    \label{fig:point-light-non-euc}
\end{figure*}
\subsection{Spotlights}
In the game, spotlights are used to represent the player's flashlight and the car's head- and tail lights.

Spotlights are modeled in the same way as point lights, with only one exception.
In the case of spotlights, we want to capture the fact that the light forms a cone.
To do that we calculate the \textit{intensity coefficient} given by
\begin{equation*}
    \mathrm{IC} = \frac{\langle l, -d \rangle - R}{r - R},
\end{equation*}
where $d$ is the vector along which the spotlight is directed, and $R$ and $r$ are the parameters defining the light cone \cite{LearnOpenGL-Light-casters}.
The intensity coefficient is then used to multiply each component of the light, giving the result shown in \autoref{fig:spotlights}.
\begin{figure*}[h]
    \centering
    \begin{subfigure}[b]{0.475\textwidth}
        \centering
        \includegraphics[width=\textwidth]{chapters/theoretical_foundations/sections/lighting/resources/player-flashlight.png}
        \caption[]%
        {{\small Player's flashlight}}
        \label{fig:spotlight-player}
    \end{subfigure}
    \hfill
    \begin{subfigure}[b]{0.475\textwidth}
        \centering
        \includegraphics[width=\textwidth]{chapters/theoretical_foundations/sections/lighting/resources/car-lights-spot.png}
        \caption[]%
        {{\small Car illumination}}
        \label{fig:spotlight-car}
    \end{subfigure}
    \caption[]
    {\small Spotlights used in the game}
    \label{fig:spotlights}
\end{figure*}
\section{Chunk worker} \label{sec:chunk-worker}
The majority of the game logic is handled by a single thread.
The only two exceptions are the physics engine (external library) and the chunk management system.
The chunk management system is split between two threads: the main thread (the thread that the rest of the application is running on) and the \textit{chunk worker's} thread.
The worker's thread is concerned with operations that are either CPU-intensive or could take a long time to execute.
More specifically, the chunk worker is performing the following operations:
\begin{enumerate}
    \item loading saved chunks from disk and generating new chunks,
    \item saving chunks to disk,
    \item updating chunks.
\end{enumerate}
The system is built around the producer-consumer pattern, i.e. the main thread communicates with the worker's thread (and \textit{vice versa}) using queues.
It's important to note that depending on the stage of an operation either thread can be the \textit{producer} or the \textit{consumer}.
We will now describe the workflow for each of the operations mentioned before.

\subsection{Loading and generating chunks}
On each render frame, the main thread calls the \texttt{UpdateCurrentPosition} method.
This method, based on the camera's current position and the render distance, determines which chunks should be loaded into the game.
If a chunk isn't already loaded or isn't queued for loading, the position of the chunk is enqueued into the \texttt{chunksToLoad} queue.

The worker thread in the \texttt{LoadChunks} function dequeues the positions of the chunks to be loaded from the disk.
If a chunk is not saved on the disk, it is generated.
The loaded/generated chunk is then enqueued into the \texttt{loaded} queue.

The main thread through the \texttt{ResolveLoaded} function dequeues a chunk and performs the following actions:
\begin{enumerate}
    \item creates a vertex array object for the chunk's mesh,
    \item creates a collision surface for the chunk,
    \item adds the chunk to the list of scene's chunks for rendering.
\end{enumerate}

It's worth noting that the operations in \texttt{ResolveLoaded} function have to be performed on the main thread because they interact with OpenGL and the physics engine APIs.
\subsection{Saving chunks}
Saving chunks is handled by a process similar to the one discussed in \autoref{subsec:loading-and-generating}.
On each frame, the main thread in the \texttt{DeleteChunks} function determines which chunks are too far from the player and thus should be removed from the game and saved to disk.
Positions of these chunks are enqueued into the \texttt{chunksToSave} queue, and their resources are freed.

The worker thread dequeues chunks' positions from the \texttt{chunksToSave} queue and saves the scalar field associated with a given chunk on the disk.
\subsection{updating chunks}
Terrain modification consists of two main steps:
\begin{enumerate}
    \item the scalar field for the modified chunk has to be modified, which involves iterating over a 3-dimensional array of numbers and modifying the values stored in that array using some function,
    \item a new mesh has to be generated based on the new values of the scalar field.
\end{enumerate}
Since the modifications happen very often \textcolor{red}{how often (each render frame or more often?? -- yeah each render frame, there's a time accumulator in place)} and the number of operations they require is rather big (of the order of the chunk size, i.e. $16^3$) it's unfeasible to do them on the main thread without severe lags.
Thus most of the work related to terrain modification is delegated to the worker thread.

The workflow for terrain modification can be described as follows.
The main thread in the \texttt{Pickaxe.ModifyTerrain} method determines which chunks are going to be affected by the terrain modification, and adds them to a buffer \texttt{buffer}.
Once all the chunks are in \texttt{buffer}, we set the \texttt{IsProcessingBatch} flag to \texttt{true} (while set to true, no further terrain modifications will be registered) and enqueue each of them into the \texttt{modificationsToPerform} queue along with some additional information (passed in the form of an instance of \texttt{ModificationArgs} struct) necessary to perform the modification.
A very important piece of information is the \texttt{batchSize} which is the size of \texttt{buffer} (its importance will become apparent later).

The worker thread dequeues chunks from the \texttt{modificationsToPerform} queue.
It modifies the scalar field using the information passed in \texttt{ModificationArgs} and generates a new mesh based on the scalar field.
Once new vertices for the mesh are calculated, it enqueues the chunk together with \texttt{batchSize} (retrieved from \texttt{modificationArgs}) into the \texttt{updatedChunks} queue.

In the \texttt{ResolveUpdated} function the main thread dequeues the \texttt{(chunk, batchSize)} pair from the \texttt{updatedChunks} queue and adds \texttt{chunk} to the \texttt{currentBatch} list.
Once the number of chunks in \texttt{currentBatch} is equal to \texttt{batchSize} of the dequeued chunk, it means that the main thread has "received back" all of the chunks from a single modification call.
The main thread then updates the GPU buffers with the new vertices of the chunk's mesh and updates the shape of the collision surface in the physics engine.
Once the whole \texttt{currentBatch} is updated, the \texttt{IsProcessingBatch} flag is set back to false.

The reason for processing chunks in batches rather than individually is simple.
If we modify chunks one by one it may be the case that when the terrain is rendered, one chunk has already been modified, while its neighbor has not, resulting in a visible gap between the two.
This problem can be seen in \autoref{fig:gaps-between-chunks} which comes from an early stage of the game's development.
\begin{figure}[h]
    \centering
    \includegraphics[width=0.8\textwidth]{chapters/system_architecture/sections/chunk_worker/resources/gaps-between-chunks.png}
    \caption{Gaps between chunks}
    \label{fig:gaps-between-chunks}
\end{figure}

\textcolor{red}{TODO: write about the zero time stuff because it may seem like we can be losing modifications f f the workewr thread is takin g unsusually long to preocess}
\chapter{Main components}
\todo{write something about the components design overview or whatever}

\section{Animator}

The \texttt{Animator} class is responsible for animating the models loaded by \texttt{ModelLoader}.
It takes animation keyframes and interpolates them using Quaternion Slerp and Vector3 Lerp which are provided by AssimpNet.
It then uses the interpolated keyframes to calculate the transformation matrices for each bone in the model which are later passed to the shader.
\input{chapters/main_components/sections/model_loader/model_loader.tex}

\subsection{\texttt{Printer} class}

The \texttt{Printer} class is responsible for printing text on the screen.
It has a static constructor in which it creates a texture with the alphabet.
To achieve that it uses the SkiaSharp library.
After the texture is created \texttt{Printer} creates an array of Rectangles that describe the position of each letter on the texture.
These rectangles are passed to the shader when the text is printed.
This way only one texture is passed to the shader and the shader uses the rectangles to get the correct letter from the texture which improves performance.
The class also provides a method for printing a letter which is used in another method to print whole strings.
There are also methods for printing strings with the specified corner at the specified position.
\subsection{\texttt{SpriteRenderer} class} \label{sprite_renderer}
\todo{this has to be rewritten because it says the same things as textures.tex}
The \texttt{SpriteRenderer} class is responsible for rendering 2D sprites.
It takes a PNG file and a JSON file as input and creates a texture and an array of rectangles that describe the position of each sprite on the texture.
It then uses the texture and the rectangles to render the sprites by passing the rectangle coordinates to the shader.
It only sets the texture once to boost performance.

An exemplary JSON file describing the position of the sprites on the texture is shown below:

\begin{verbatim}
    {
      "width": 10,
      "height": 10,
      "items": [
        {
          "name": "someItem",
          "x": 2,
          "y": 3,
          "width": 2,
          "height": 1
        },
        \dots
      ]
    }
\end{verbatim}


The width and height describe the size of the sprite sheet.
The \texttt{x,y} coordinates describe the position of the sprite on the sprite sheet.
The width and height describe the size of the sprite.
The name of the item is used to identify the sprite.

\todo{Screenshots? FPS, position, ...}
\section{Menu}
The \texttt{Menu} class is responsible for rendering the menu and handling user input.
It makes use of \texttt{SpriteRenderer} to render buttons and graphics.
It makes use of the \texttt{Printer} class to render text.
The menu is described in more detail in \autoref{subsec:menu}.

\section{Chunk Generator} \label{chunk_generator}
\todo{we should describe that the scalar field generation is different for spherical space ("stitching the spheres")}
The Chunk Generator component is responsible for procedural terrain generation.
The generation process is encapsulated inside the \texttt{ChunkFactory} class.

A chunk is generated in two main steps:
\begin{enumerate}
    \item A scalar field of a given size is created.
          The values of the scalar field are generated based on the values of Perlin noise.
          This step is performed by the \texttt{ScalarFieldGenerator} class.
    \item The marching cubes algorithm is used to create a mesh;
          positions and normal vectors of the mesh's vertices are obtained in this step using the \texttt{MeshGenerator} class.
\end{enumerate}
For more information on terrain generation, refer to \autoref{sec:implementation_terrain}.
\section{Chunk worker} \label{sec:chunk-worker}
The majority of the game logic is handled by a single thread.
The only two exceptions are the physics engine (external library) and the chunk management system.
The chunk management system is split between two threads: the main thread (the thread that the rest of the application is running on) and the \textit{chunk worker's} thread.
The worker's thread is concerned with operations that are either CPU-intensive or could take a long time to execute.
More specifically, the chunk worker is performing the following operations:
\begin{enumerate}
    \item loading saved chunks from disk and generating new chunks,
    \item saving chunks to disk,
    \item updating chunks.
\end{enumerate}
The system is built around the producer-consumer pattern, i.e. the main thread communicates with the worker's thread (and \textit{vice versa}) using queues.
It's important to note that depending on the stage of an operation either thread can be the \textit{producer} or the \textit{consumer}.
We will now describe the workflow for each of the operations mentioned before.

\subsection{Loading and generating chunks}
On each render frame, the main thread calls the \texttt{UpdateCurrentPosition} method.
This method, based on the camera's current position and the render distance, determines which chunks should be loaded into the game.
If a chunk isn't already loaded or isn't queued for loading, the position of the chunk is enqueued into the \texttt{chunksToLoad} queue.

The worker thread in the \texttt{LoadChunks} function dequeues the positions of the chunks to be loaded from the disk.
If a chunk is not saved on the disk, it is generated.
The loaded/generated chunk is then enqueued into the \texttt{loaded} queue.

The main thread through the \texttt{ResolveLoaded} function dequeues a chunk and performs the following actions:
\begin{enumerate}
    \item creates a vertex array object for the chunk's mesh,
    \item creates a collision surface for the chunk,
    \item adds the chunk to the list of scene's chunks for rendering.
\end{enumerate}

It's worth noting that the operations in \texttt{ResolveLoaded} function have to be performed on the main thread because they interact with OpenGL and the physics engine APIs.
\subsection{Saving chunks}
Saving chunks is handled by a process similar to the one discussed in \autoref{subsec:loading-and-generating}.
On each frame, the main thread in the \texttt{DeleteChunks} function determines which chunks are too far from the player and thus should be removed from the game and saved to disk.
Positions of these chunks are enqueued into the \texttt{chunksToSave} queue, and their resources are freed.

The worker thread dequeues chunks' positions from the \texttt{chunksToSave} queue and saves the scalar field associated with a given chunk on the disk.
\subsection{updating chunks}
Terrain modification consists of two main steps:
\begin{enumerate}
    \item the scalar field for the modified chunk has to be modified, which involves iterating over a 3-dimensional array of numbers and modifying the values stored in that array using some function,
    \item a new mesh has to be generated based on the new values of the scalar field.
\end{enumerate}
Since the modifications happen very often \textcolor{red}{how often (each render frame or more often?? -- yeah each render frame, there's a time accumulator in place)} and the number of operations they require is rather big (of the order of the chunk size, i.e. $16^3$) it's unfeasible to do them on the main thread without severe lags.
Thus most of the work related to terrain modification is delegated to the worker thread.

The workflow for terrain modification can be described as follows.
The main thread in the \texttt{Pickaxe.ModifyTerrain} method determines which chunks are going to be affected by the terrain modification, and adds them to a buffer \texttt{buffer}.
Once all the chunks are in \texttt{buffer}, we set the \texttt{IsProcessingBatch} flag to \texttt{true} (while set to true, no further terrain modifications will be registered) and enqueue each of them into the \texttt{modificationsToPerform} queue along with some additional information (passed in the form of an instance of \texttt{ModificationArgs} struct) necessary to perform the modification.
A very important piece of information is the \texttt{batchSize} which is the size of \texttt{buffer} (its importance will become apparent later).

The worker thread dequeues chunks from the \texttt{modificationsToPerform} queue.
It modifies the scalar field using the information passed in \texttt{ModificationArgs} and generates a new mesh based on the scalar field.
Once new vertices for the mesh are calculated, it enqueues the chunk together with \texttt{batchSize} (retrieved from \texttt{modificationArgs}) into the \texttt{updatedChunks} queue.

In the \texttt{ResolveUpdated} function the main thread dequeues the \texttt{(chunk, batchSize)} pair from the \texttt{updatedChunks} queue and adds \texttt{chunk} to the \texttt{currentBatch} list.
Once the number of chunks in \texttt{currentBatch} is equal to \texttt{batchSize} of the dequeued chunk, it means that the main thread has "received back" all of the chunks from a single modification call.
The main thread then updates the GPU buffers with the new vertices of the chunk's mesh and updates the shape of the collision surface in the physics engine.
Once the whole \texttt{currentBatch} is updated, the \texttt{IsProcessingBatch} flag is set back to false.

The reason for processing chunks in batches rather than individually is simple.
If we modify chunks one by one it may be the case that when the terrain is rendered, one chunk has already been modified, while its neighbor has not, resulting in a visible gap between the two.
This problem can be seen in \autoref{fig:gaps-between-chunks} which comes from an early stage of the game's development.
\begin{figure}[h]
    \centering
    \includegraphics[width=0.8\textwidth]{chapters/system_architecture/sections/chunk_worker/resources/gaps-between-chunks.png}
    \caption{Gaps between chunks}
    \label{fig:gaps-between-chunks}
\end{figure}

\textcolor{red}{TODO: write about the zero time stuff because it may seem like we can be losing modifications f f the workewr thread is takin g unsusually long to preocess}

\section{Physics}
The Physics component is responsible for collision detection, ray casting, and physical modeling.
Each game object is represented by a \textit{body} (each body has an associated \textit{body handle}) in the physics engine.
This one-to-one mapping is stored in the \texttt{Scene} class in an object of \texttt{SimulationMembers} type.
\texttt{SimulationMembers} allows for adding and removing simulation members as well as accessing handles to bodies in the physics engine associated with a given game object.
Game objects are represented by either simple or composite shapes in the physical simulation.
Simple shapes used in the game include cylinders, capsules, and boxes.
Composite shapes arise by composing simple shapes.
To facilitate debugging, the Physics component also provides a way to extract all shapes from the simulation (and decompose composite shapes) which then can be shown in debug mode.
The body that represents the terrain was created as a separate shape from the terrain mesh triangle-by-triangle.

Game objects can listen and respond to collisions by registering their contact callbacks via the API in \texttt{SimulationManager}.
Such objects implement the \texttt{IContactEventListener} interface whose implementations define the contact callbacks.
The only information about the collision is which two bodies collided and where.

Some game objects can cast rays.
An example of such an object is the player who uses a ray to define a place where terrain modification should take place.
Each ray in the physics engine has an ID number, direction, etc.
An object that wishes to cast rays has to therefore implement an interface (\texttt{IRayCaster}) that exposes all the necessary information.

\section{Scene}

The \texttt{Scene} class is responsible for storing all the objects in the game the player can interact with.
It is also responsible for disposing of them.
Here is the full list of objects stored in a \texttt{Scene} instance:

\begin{itemize}
  \item Chunks
  \item LightSources
  \item Projectiles
  \item Bots
  \item Cars
  \item Player
  \item Camera
  \item SimulationMembers
  \item SimulationManager
\end{itemize}


\section{Controllers}

There are eight controller classes in the project:

\begin{itemize}
  \item  \texttt{PlayerController}
  \item \texttt{BotsController} \todo{Not sure if bots logic is described anywhere, we can give a description here}
  \item \texttt{ChunksController}
  \item \texttt{ProjectilesController}
  \item \texttt{VehiclesController}
  \item \texttt{LightSourcesController}
  \item \texttt{HudController}
  \item \texttt{BoundingShapesController}
\end{itemize}

All the controller classes serve the same purpose.
They exist to render objects and deal with object callbacks.
For example, \texttt{ProjectilesController} renders all existing projectiles and also removes the dead projectiles from the scene.
Each projectile has a \texttt{IsAlive} property which is read every time a render frame callback is called by the controller to check if the projectile is still alive.

\section{Context}

The Context component is responsible for input handling and performing actions on each frame.
There are two sources of user input considered in the video game: keyboard and mouse.
Keys and mouse buttons can be in one of three states: \textit{pressed}, \textit{down}, and \textit{up}.
Additionally, the mouse can also generate events when it's moved.

The \texttt{Context} class stores mappings between different types of input and actions that should be triggered by the given input.
Changes in the input state are tracked by OpenTK and the \texttt{Context} activates relevant actions as a response.
It is a convention that every class that registers new actions in the \texttt{Context} implements the \texttt{IInputSubscriber} interface.
Also, it is worth noting that, for better performance, the Context will only trigger actions associated with the input that has itself been previously registered.
\section{Transporter}
The Transporter component is vital for the spherical geometry mode of the game.
It is responsible for transporting game objects between spheres.
All transporters implement the \texttt{ITransporter} interface.

Each game object has a property \texttt{CurrentSphereId} that attains one of two values: 0 or 1 depending on which sphere the object is currently in (this property is a part of \texttt{ISimulationMember} interface).
Whenever an object moves, the Transporter determines whether it should be transported to the opposing sphere by calculating the object's distance from its current sphere's center.

The transportation process changes the object's position and velocity.
In the case of objects modeled as compound bodies, it is necessary to alter the positions and velocities of each of the components.
When a camera is attached to an object that gets transported, it also has to be updated accordingly.
More specifically, the Transporter transforms the camera's front vector and updates the information about the camera's current sphere.

The functionalities mentioned above are only relevant in the spherical geometry mode.
To make the system design consistent across all modes, there is also a \texttt{NullTransporter} implementation of the \texttt{ITransporter} interface which is used in hyperbolic and Euclidean geometry modes.
This is a dummy class with methods that don't do anything.

\section{Bits management}
\todo{Not sure if bots logic is described anywhere, we can give a description here}
\chapter{User Manual}\label{ch:user_manual}
This section describes how to launch and later use the application.
The user controls were made with industry standards in mind, so the user should not have any problems with them.
However, if the user does not have much experience with video games, this section will help them get started as it provides a detailed start-to-finish description of how to play the game.

\subsubsection{Launching the game}
The application starts in the main menu as shown in \autoref{fig:main_menu}.
Here you can start a new game or load a previously saved one.
You can also delete previous saves, view controls, and exit the game.
To start a game for the first time, you need to click on the "New Game" button which will show the "New Game Screen".

You can show/hide the main menu by pressing the \keys{\escwin} key when a game is running.
While the game is running the main menu will also show a "Resume" button which will resume the game and hide the menu.

\begin{figure}[H]
    \centering
    \includegraphics[width=0.8\textwidth]{chapters/user_manual/resources/main-menu.png}
    \caption{Main menu}
    \label{fig:main_menu}
\end{figure}
\section{New Game Screen}
The game can be started from the New Game screen shown in \autoref{fig:new_game}.
To start a game, the user has to click on the "Input Game Name" input text box.
This will allow them to enter a name for the game.
If there is no game save with the same name, the text box will light up green.
If there is already a game save with the same name, the text box will light up red.
If the box is green, the user can press one of the buttons to start a new game in the chosen geometry.

\begin{figure}[h]
    \centering
    \includegraphics[width=0.8\textwidth]{chapters/user_manual/resources/new-game-no-input.png}
    \caption{New Game screen}
    \label{fig:new_game}
\end{figure}
\subsubsection{Load Game Screen}
The game can be loaded from the Load Game screen as shown in \autoref{fig:load_game}.
This screen displays your 9 most recent saves.
You can click on one of the saves to load it.
If you want to load a save that is older than that you have to delete your more recent saves using the "Delete Game" described in \autoref{delete_save_screen}

\begin{figure}[H]
    \centering
    \includegraphics[width=0.8\textwidth]{chapters/user_manual/resources/load-game.png}
    \caption{Load Game screen}
    \label{fig:load_game}
\end{figure}
\subsubsection{Delete Save Screen} \label{delete_save_screen}
The "Delete Game" screen shown in \autoref{fig:delete_save} allows you to delete your saves.
It will display your 9 most recent saves.
To delete a save click on its name.
If a game is currently running you cannot delete its save.

\begin{figure}[H]
    \centering
    \includegraphics[width=0.8\textwidth]{chapters/user_manual/resources/delete-game.png}
    \caption{Delete Save screen}
    \label{fig:delete_save}
\end{figure}
\section{Controls}
The controls for the game are presented in the \autoref{tab:game_key_functions}.
If the user forgets them they can at any time press the "Controls" button in the menu to see them in game as shown in Figure \ref{fig:controls}.

\begin{table}[h]
    \centering
    \begin{tabular}{|m{3cm}|m{8cm}|}
        \hline
        \textbf{Key}              & \textbf{Function}                                             \\
        \hline
        \keys{W}                  & Move Forward                                                  \\
        \hline
        \keys{S}                  & Move Back                                                     \\
        \hline
        \keys{A}                  & Move Left                                                     \\
        \hline
        \keys{D}                  & Move Right                                                    \\
        \hline
        \keys{\shift}             & Sprint                                                        \\
        \hline
        \keys{\SPACE}             & Jump                                                          \\
        \hline
        \texttt{Left Mouse}       & Use Item                                                      \\
        \hline
        \texttt{Right Mouse}      & Use Item's second ability                                     \\
        \hline
        \keys{\escwin}            & Show/Hide Menu                                                \\
        \hline
        \keys{\tab}               & Switch Camera between 1st and 3rd person                      \\
        \hline
        \texttt{Scroll}           & Change FOV                                                    \\
        \hline
        \keys{0} through \keys{9} & Select Item                                                   \\
        \hline
        \keys{C}                  & Enter Car                                                     \\
        \hline
        \keys{F}                  & Flip Car (only works outside the car)                         \\
        \hline
        \keys{L}                  & Leave Car                                                     \\
        \hline
        \keys{Y}                  & Toggle Flashlight/Toggle car reflectors (when inside the car) \\
        \hline
    \end{tabular}
    \caption{Keyboard Key Functions for Game Controls}
    \label{tab:game_key_functions}
\end{table}

\begin{figure}[H]
    \centering
    \includegraphics[width=0.8\textwidth]{chapters/user_manual/resources/controls.png}
    \caption{Controls screen}
    \label{fig:controls}
\end{figure}
\section{Items}
Items are a big part of the game.
Different items in the game have different uses.
Description of all the in-game items can be found in \autoref{tab:mytable}.
To use the items the player has to first select the desired item using number keys \keys{0} through \keys{9}.
The selected item will be highlighted in the inventory as shown in \autoref{fig:inventory}; in this example, the selected item is the gun.
The player can use the item by pressing the left mouse button (\texttt{LMB}) or the right mouse button (\texttt{RMB}).
The effect of the item depends on the item itself and the mouse button used.

\begin{table}[!htb]
    \centering
    \begin{tabular}{|c|p{5cm}|p{5cm}|}
        \hline
        Item                                                                         & \texttt{LMB} Effect                                                                   & \texttt{RMB} Effect \\
        \hline
        \includegraphics[width=1cm]{chapters/user_manual/resources/bullet.png}       & No effect                                                                             & No effect           \\
        \hline
        \includegraphics[width=1cm]{chapters/user_manual/resources/pistol.png}       & Fires a bullet if there is one in the inventory. Removes a bullet from the inventory. & No effect           \\
        \hline
        \includegraphics[width=1cm]{chapters/user_manual/resources/pickaxe-slow.png} & Mines slowly.                                                                         & Builds slowly.      \\
        \hline
        \includegraphics[width=1cm]{chapters/user_manual/resources/pickaxe-mid.png}  & Mines.                                                                                & Builds.             \\
        \hline
        \includegraphics[width=1cm]{chapters/user_manual/resources/pickaxe-fast.png} & Mines quickly.                                                                        & Builds quickly.     \\
        \hline
    \end{tabular}
    \caption{Table of in-game items}
    \label{tab:mytable}
\end{table}

\begin{figure}[!htb]
    \centering
    \includegraphics[width=0.8\textwidth]{chapters/user_manual/resources/inventory.png}
    \caption{Inventory}
    \label{fig:inventory}
\end{figure}
\section{Gameplay}\label{sec:gameplay}
After the game is started, the user can finally start playing.
At the start, the screen will look like in \autoref{fig:gameplay}.
The user can see their position in the top left corner, the current frames per second in the top right corner, and the inventory in the bottom of the screen.
In the middle of the screen, there is a crosshair that shows where the user is aiming.
The user can also see their character model.
By using the controls described in \autoref{sec:controls}, the user can start moving around and exploring the world.


\begin{figure}[!htb]
    \centering
    \includegraphics[width=0.8\textwidth]{chapters/user_manual/resources/gameplay.png}
    \caption{Controls screen}
    \label{fig:gameplay}
\end{figure}
\chapter{Functionalities}
\textcolor{red}{This chapter should have a different folder name and the tex file should also be renamed.
    Maybe we could split the thesis into chapters like Theoretical foundations, System architecture, Functionalities/Features/??? (move the terrain stuff here as well), ...}
\section{Environment}
Even though the game can be played in non-Euclidean spaces which makes it inherently unrealistic, we decided to add some elements that would make the scenes portrayed in the game feel familiar.
One such element is the Earth-like terrain, described in detail in the previous sections.
Another property of the real world that we wanted to capture in the game was the daytime cycle.
In the game, the full cycle is 10 minutes long, with 5 minutes long daytime and 5 minutes long nighttime.
We also added transitions between day and night to give the Earth-like experience of sunrise and sunset.
The scene during various times of the day is shown in \autoref{fig:cycle}.

\begin{figure*}[h]
    \centering
    \begin{subfigure}[b]{0.475\textwidth}
        \centering
        \includegraphics[width=\textwidth]{chapters/lighting/sections/environment/resources/day.png}
        \caption[]%
        {{\small Day}}
        \label{fig:cycle-day}
    \end{subfigure}
    \hfill
    \begin{subfigure}[b]{0.475\textwidth}
        \centering
        \includegraphics[width=\textwidth]{chapters/lighting/sections/environment/resources/night.png}
        \caption[]%
        {{\small Night}}
        \label{fig:cycle-night}
    \end{subfigure}
    \vskip\baselineskip
    \begin{subfigure}[b]{0.475\textwidth}
        \centering
        \includegraphics[width=\textwidth]{chapters/lighting/sections/environment/resources/sunrise.png}
        \caption[]%
        {{\small Sunrise}}
        \label{fig:cycle-sunrise}
    \end{subfigure}
    \hfill
    \begin{subfigure}[b]{0.475\textwidth}
        \centering
        \includegraphics[width=\textwidth]{chapters/lighting/sections/environment/resources/sunset.png}
        \caption[]%
        {{\small Sunset}}
        \label{fig:cycle-sunset}
    \end{subfigure}
    \caption[]
    {\small Day night cycle in the game}
    \label{fig:cycle}
\end{figure*}

The implementation of the day-night cycle relies on two components: directional lighting (described in \autoref{subsec:directional-lighting}) which corresponds to the light coming from the sun and a \textit{skybox} representing the sky.
Conceptually, a skybox is a cube made out of 6 images, one per each side, that encompasses the scene thus creating an illusion that the world is much bigger than it is in reality.
A skybox can be implemented in OpenGL using a special type of texture, a \textit{cubemap}, i.e. a texture that contains 6 individual 2D textures.
In the game, we're using images of the night sky obtained from an HDR file \url{https://www.reddit.com/r/blender/comments/3ebzwz/free_space_hdrs_1/} using an online utility program \url{https://matheowis.github.io/HDRI-to-CubeMap/}.
The images were then slightly edited to make the stars appear larger. \question{Can we use and edit images found online? Both in the project and in the document?}
The vertices of the cube passed to the vertex shader are transformed using the model, view, and projection matrices.
The model matrix is responsible for rotating the skybox (similarily to how stars appear to move across the night sky as the Earth is rotating).
The view matrix has to be modified so that the skybox doesn't move along with the camera.
The part responsible for translation can be removed from the view matrix by replacing the last row the the view matrix with the vector $[0, 0, 0, 1]$ \cite{LearnOpenGL-Cubemaps}.

The day is split into \textit{phases}, each characterized by the color of the sky at the zenith, the sky's color at the horizon, the color of the sun, and \textit{stars' visibility factor}.
In the fragment shader, we determine the color of each fragment.
This is done by first obtaining the zenith and horizon colors by interpolating between the corresponding colors for the previous and next phase based on the current time.
In the same manner, we obtain the current stars' visibility factor.
Then, we obtain a sky color for a given fragment by interpolating between the zenith and horizon colors based on the height of the fragment\footnote{The "position" of a fragment in this context is given by world space coordinates, normalized so that we're treating the points as located on a sphere (\textit{skydome}). The height is then simply the $y$ coordinate of the fragment's position.}.
The sun's position is given by a vector $s$ that rotates at the same rate as the skybox.
Calculating a dot product $d$ of $s$ with the current fragment's position allows us to easily draw the sun and the sun glare by mixing the sky's color with the sun's color in proportions depending on $d$.
As the last step, we mix the pure-day-time color of the sky with the pure-night-time texture of the stars in proportions given by the stars' visibility factor.

The day-night cycle hasn't been implemented for the spherical space, as the terrain in the spherical space fully "encloses" the scene leaving no way of seeing anything "outside".
\section{Lighting}
Lighting is an important aspect of the game, it makes the game more immersive and has a major impact on how the player perceives the game.
Artificial light sources present in the game are also indispensable when exploring the world during the in-game night.

\subsection{Directional lighting} \label{subsec:directional-lighting}
\chapter{Results}
\todo{Add some introduction.}

\section{Performance}
The performance and resource utilization of the game has been tested using Visual Studio's performance profiler\footnote{\url{https://learn.microsoft.com/en-us/visualstudio/profiling/?view=vs-2022}}.
The analysis was focused on three aspects:
\begin{enumerate}
    \item File I/O,
    \item CPU usage,
    \item memory usage.
\end{enumerate}
The data was collected on a machine running on x64-based Windows 11 with Intel(R) Core(TM) i7-9750H CPU, 32 GB of RAM, and NVIDIA GeForce GTX 1650 GPU.
During the first minute of the data collection, the player was running constantly, approximately in one direction, while modifying the terrain at the same time.
Starting at the 1-minute mark, the player turned around and started running in the opposite direction, still modifying the terrain, and also shooting at the bots from time to time.
The results of the profiling are shown in \autoref{fig:diag-session}.
\begin{figure}[h]
    \centering
    \includegraphics[width=1\textwidth]{chapters/results/sections/performance/resources/diag-session.png}
    \caption{Resource utilization of the game}
    \label{fig:diag-session}
\end{figure}

The "File Reads" graph shows the amount of data read in MB.
The initial spike in the file reads corresponds to the game reading the shader files and PNG files with textures.
It can be seen that starting at the 1-minute mark, there are frequent file reads.
This is because the chunks that were generated (and subsequently removed from the game and saved) during the initial run in one direction are now being revisited by the player and read from the disk.

The "Process Memory" graph shows that the memory usage remains approximately constant at around 350 MB.

The "CPU" graph shows that the CPU utilization peaks at around 10\%.

The total amount of disk space required to store the game saves for this session was 100 MB.
\section{Gameplay}
\textit{Hyper} was designed as an \textit{open world} game.
In a game of this type, the player is not constrained to achieving a specific goal and has a large degree of freedom to explore, interact with, or modify the game environment \cite{Open-World-MW}.
In this section, we'll show the various aspects of this concept in our game.

\subsection{Terrain editing}
The player has three "terrain modifiers", represented with pickaxe symbols, at their disposal.
Using the terrain modifier, the player can edit the game's landscape by building new structures or digging in the ground.
As an example of the "creational capabilities", we show in \autoref{fig:hyper-logo} how the letters making up the game's name could be created inside the game.
\begin{figure}[h]
    \centering
    \includegraphics[width=0.8\textwidth]{chapters/results/sections/gameplay/resources/hyper-logo-night-2.png}
    \caption{The letters of the word "Hyper" created in the game}
    \label{fig:hyper-logo}
\end{figure}

As mentioned before the terrain modifiers can also be used for carving in the terrain.
To illustrate that, in \autoref{fig:tunnel-under-hill} we show a tunnel that was dug through a hill.
\begin{figure}[h]
    \centering
    \includegraphics[width=0.8\textwidth]{chapters/results/sections/gameplay/resources/tunnel-with-car.png}
    \caption{Tunnel dug under a hill}
    \label{fig:tunnel-under-hill}
\end{figure}

% TODO: move this part to discussion/problems
% \todo{I don't know if we should be writing that here or at all.}
% It's important to note that modifying the terrain below a certain level can become "slow".
% This is because the scalar field attains larger values for points with small values of the $y$ coordinate.
% Additionally, the process of terrain modification may appear slightly laggy in spherical space due to bigger chunks' sizes.


\subsection{Exploring the game world}
The game starts in one of the various "landscapes" such as a desert, a forest, etc. each characterized by different terrain generation parameters and colors.
Some of the possible terrains are shown in \todo{add screenshots}.

To make exploring the game world easier, the player can get into a car that moves considerably faster than the player.
\autoref{fig:car-in-hyperbolic} shows the car driving in hyperbolic space.
\begin{figure}[!htb]
    \centering
    \includegraphics[width=0.8\textwidth]{chapters/results/sections/gameplay/resources/car-in-hyperbolic.png}
    \caption{Riding a car in hyperbolic space}
    \label{fig:car-in-hyperbolic}
\end{figure}
Even though the car is much faster than the player it can roll over when traversing a particularly bumpy terrain.
For this reason, we included an option to flip the car back on its wheels when that happens.
\subsection{Interacting with the world's inhabitants}
To make the world more interactive we decided to populate it with NPCs also called bots.
Bots can be either hostile toward the player or neutral.
Hostile bots shoot projectiles and walk toward the player once they get into the player's proximity.
A group of bots shooting projectiles at the player in spherical space is shown in \autoref{fig:firing-squad}.
\begin{figure}[h]
    \centering
    \includegraphics[width=0.8\textwidth]{chapters/results/sections/gameplay/resources/firing-squad.png}
    \caption{Fighting with bots}
    \label{fig:firing-squad}
\end{figure}
The neutral bots are unbothered by the player's presence and just walk around.
This can change, however.
Once the player shoots at a neutral bot, it'll become hostile.
\section{Depicting non-Euclidean spaces}
One of the main features of the game is the ability to explore non-Euclidean spaces.
To assess our depictions of non-Euclidean spaces, we decided to compare them with the ones from the game \textit{Hyperbolica}\footnote{\url{https://codeparade.itch.io/hyperbolica}} by \textit{CodeParade}.
It should be noted that the approach used in our game, although relatively simple makes it impossible to capture some of the interesting properties of non-Euclidean geometry, like tiling the plane with right-angled regular pentagons in hyperbolic space.

\subsection{Hyperbolic space}
From the visual standpoint, the hyperbolic space can be identified by the fact that the otherwise flat terrain appears "curved downward".
This may create an illusion that the terrain is wrapped around a giant sphere.
\todo{That's at least how it looks to me}
However, by exploring the hyperbolic space, we can quickly notice that it is infinite, just like the Euclidean space.
\autoref{fig:hyperbolic-space-games} shows the comparison of the depictions of hyperbolic space between our game and \textit{Hyperbolica}.
\begin{figure*}[h]
    \centering
    \begin{subfigure}[b]{0.475\textwidth}
        \centering
        \includegraphics[width=\textwidth]{chapters/results/sections/non_euclidean/resources/hyperbolic-in-hyper.png}
        \caption[]%
        {{\small \textit{Hyper}}}
        \label{fig:hyperbolic-space-games-hyper}
    \end{subfigure}
    \hfill
    \begin{subfigure}[b]{0.475\textwidth}
        \centering
        \includegraphics[width=\textwidth]{chapters/results/sections/non_euclidean/resources/hyperbolica-hyperbolic.png}
        \caption[]%
        {{\small \textit{Hyperbolica \cite{Hyperbolica-Hyperbolic}}}}
        \label{fig:hyperbolic-space-games-hyperbolica}
    \end{subfigure}
    \caption[]
    {\small Hyperbolic space}
    \label{fig:hyperbolic-space-games}
\end{figure*}

\todo{We have to mention that our non-Euclidean spaces are "fake" in the sense that you don't have 5-sided right pentagons and all that jazz.}
\subsection{Spherical space}
Playing the video game in spherical space gives the impression that the terrain is wallpapered onto the inside of a giant sphere.
Unlike the hyperbolic space, the spherical space is finite.
Comparison in \autoref{fig:spherical-space-games} shows that our implementation provides visual effects to a certain degree similar to those in \textit{Hyperbolica}.
\begin{figure*}[!htb]
    \centering
    \begin{subfigure}[b]{0.475\textwidth}
        \centering
        \includegraphics[width=\textwidth]{chapters/results/sections/non_euclidean/resources/spherical-in-hyper.png}
        \caption[]%
        {{\small \textit{Hyper}}}
        \label{fig:spherical-space-games-hyper}
    \end{subfigure}
    \hfill
    \begin{subfigure}[b]{0.5\textwidth}
        \centering
        \includegraphics[width=\textwidth]{chapters/results/sections/non_euclidean/resources/hyperbolica-1.png}
        \caption[]%
        {{\small \textit{Hyperbolica \cite{Hyperbolica-Spherical}}}}
        \label{fig:spherical-space-games-hyperbolica}
    \end{subfigure}
    \caption[]
    {\small Spherical space}
    \label{fig:spherical-space-games}
\end{figure*}

The effect of characters bending increasingly as the player gets further from them is another example of an unusual visual effect in spherical space.
Furthermore, unlike in Euclidean geometry, objects further from the player do not always appear smaller, sometimes they can even appear larger.
This effect of "reversed perspective" can be seen in both games as shown in \autoref{fig:spherical-space-reversed-perspective}.
In \autoref{fig:spherical-space-reversed-perspective-hyper}, the car and the bot to the right of the car's hood are further from the camera than other objects in the scene.
\begin{figure*}[!htb]
    \centering
    \begin{subfigure}[b]{0.475\textwidth}
        \centering
        \includegraphics[width=\textwidth]{chapters/results/sections/non_euclidean/resources/spherical-car.png}
        \caption[]%
        {{\small \textit{Hyper}}}
        \label{fig:spherical-space-reversed-perspective-hyper}
    \end{subfigure}
    \hfill
    \begin{subfigure}[b]{0.5\textwidth}
        \centering
        \includegraphics[width=\textwidth]{chapters/results/sections/non_euclidean/resources/hyperbolica-2.png}
        \caption[]%
        {{\small \textit{Hyperbolica}}}
        \label{fig:spherical-space-reversed-perspective-hyperbolica}
    \end{subfigure}
    \caption[]
    {\small Reversed perspective}
    \label{fig:spherical-space-reversed-perspective}
\end{figure*}
It's also important to note that bullets no longer fly in straight lines and that their trajectory is bent instead.
By playing the game it can be experienced that steering the car or moving the player's character appears unintuitive close to either of the spheres' boundaries.
\input{chapters/discussion_and_conclusions/discussion.tex}

\todo{Add chapter with results (what we accomplished, what we failed to accomplish, plans for future development, etc.)}

% ------------------------------- BIBLIOGRAPHY ---------------------------
% LEXICOGRAPHICAL ORDER BY AUTHORS' LAST NAMES
% FOR AMBITIOUS ONES - USE BIBTEX


\printbibliography
\pagenumbering{gobble}
\thispagestyle{empty}



% ----------------------- LIST OF SYMBOLS AND ABBREVIATIONS ------------------
\chapter*{List of symbols and abbreviations}

\begin{tabular}{cl}
  nzw.           & nadzwyczajny  \\
  *              & star operator \\
  $\widetilde{}$ & tilde
\end{tabular}
\\
If you don't need it, delete it.
\thispagestyle{empty}


% ----------------------------  LIST OF FIGURES --------------------------------
\listoffigures
\thispagestyle{empty}
If you don't need it, delete it.


% -----------------------------  LIST OF TABLES --------------------------------
\listoftables
\thispagestyle{empty}
If you don't need it, delete it.

% -----------------------------  LIST OF APPENDICES ---------------------------
\chapter*{List of appendices}
\begin{enumerate}
  \item Appendix 1
  \item Appendix 2
  \item In case of no appendices, delete this part.
\end{enumerate}
\thispagestyle{empty}


\end{document}
