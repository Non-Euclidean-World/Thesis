{  \fontsize{12}{14} \selectfont
\begin{abstract}

    \begin{center}
        \title
    \end{center}
    Procedural generation, as a method used in creating video games, has experienced a significant increase in popularity in recent years.
    It has been employed extensively in acclaimed titles such as \textit{Minecraft} and \textit{No Man's Sky} to create virtually infinite worlds that the player is free to interact with.
    On the other hand, a recent game \textit{Hyperbolica} popularized a novel idea of making the virtual worlds even more interesting by setting them in non-Euclidean spaces.
    This work describes the implementation of a video game that incorporates both of the aforementioned concepts.

    In the game a technique of transforming Euclidean space into a hyperbolic or spherical space was used, which allows for the use of established algorithms.
    This approach simplifies the implementation of the game logic in different geometries as it allows sharing large parts of the implementation across the three geometries.
    It also allows for changing the degree to which the scene is visually curved in hyperbolic geometry at runtime, which is the best way to showcase the differences between the geometries.

    The game also features a number of classic game mechanics, such as terrain editing, character animations, day and night cycle, and many more.
    As a result, the game offers a unique experience of exploring and interacting with worlds inherently different from the one we live in. \\

    \noindent \textbf{Keywords:} Non-Euclidean Geometry, Hyperbolic Geometry, Spherical Geometry, Procedural Generation, Video Games, OpenGL, C\#
    \question{Should keywords start with capital letters?}
\end{abstract}
}