{  \fontsize{12}{14} \selectfont
\begin{abstract}

    \begin{center}
        \title
    \end{center}
    Procedural generation, as a method used in creating video games, has experienced a significant increase in popularity in recent years.
    It has been employed extensively in acclaimed titles such as \textit{Minecraft} and \textit{No Man's Sky} to create virtually infinite worlds that the player is free to interact with.
    On the other hand, a recent game \textit{Hyperbolica} popularized a novel idea of making the virtual worlds even more interesting by setting them in non-Euclidean spaces.
    This work describes the implementation of a video game that incorporates both of the aforementioned concepts.

    In the game, we use and expand on the technique of transforming Euclidean space into hyperbolic or spherical space introduced by László Szirmay-Kalos and Milán Magdics in \cite{Szirmay-Kalos2022}.
    The approach allows for the use of established algorithms, developed with Euclidean geometry in mind, in other geometries.
    %This approach also simplifies the implementation of the game logic in different geometries as large parts of the implementation can be shared across the three geometries.
    Furthermore, core parts of the game logic can be shared between different geometries, which vastly simplifies the implementation.
    The approach also allows for changing the degree to which the scene is visually curved in hyperbolic geometry at runtime.

    The procedural terrain generation in the game is based on the marching cubes algorithm with Perlin noise used for generating the underlying scalar field.

    The game also includes several classic game features, such as terrain editing, character animations, day/night cycle, and many more.
    As a result, the game offers a unique experience of exploring and interacting with worlds inherently different from the one we live in. \\

    \noindent \textbf{Keywords:} Non-Euclidean Geometry, Hyperbolic Geometry, Spherical Geometry, Procedural Generation, Video Games, OpenGL, C\#
    \question{Should keywords start with capital letters?}
\end{abstract}
}