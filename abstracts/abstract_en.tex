{  \fontsize{12}{14} \selectfont
\begin{abstract}

    \begin{center}
        \title
    \end{center}
    Procedural generation, as a method used in creating video games, has experienced a significant increase in popularity in recent years.
    It has been employed extensively in acclaimed titles such as \textit{Minecraft} and \textit{No Man's Sky} to create virtually infinite worlds that the player is free to interact with.
    On the other hand, a recent game \textit{Hyperbolica} popularized a novel idea of making the virtual worlds even more interesting by setting them in non-Euclidean spaces.
    The objective of this thesis is to create a small video game incorporating both of the aforementioned concepts.
    This work describes the implementation of a video game that incorporates both of the aforementioned concepts.

    The game transforms Euclidean space into a hyperbolic or spherical space, which allows for the use of established algorithms.
    This approach simplifies the implementation of the game logic in different geometries.
    It also allows for changing the curvature of the space at runtime, which is the best way to showcase the differences between the geometries.
    This approach also could theoretically be used to create a game that takes place in a world that is a combination of different geometries.

    The game also features a number of classic game mechanics, such as terrain editing, character animations, day and night cycle, and many more.
    This makes learning about the non-Euclidean geometry more fun and engaging.

    \noindent \textbf{Keywords:} Non-Euclidean Geometry, Hyperbolic Geometry, Spherical Geometry, Procedural Generation, Video Games, OpenGL, C\#
\end{abstract}
}