{  \fontsize{12}{14} \selectfont
\begin{abstract}

    \begin{center}
        \title
    \end{center}
    Procedural generation, as a method used in creating video games, has experienced a significant increase in popularity in recent years.
    It has been employed extensively in acclaimed titles such as \textit{Minecraft} and \textit{No Man's Sky} to create virtually infinite worlds that the player is free to interact with.
    On the other hand, a recent game \textit{Hyperbolica} popularized a novel idea of making the virtual worlds even more interesting by setting them in non-Euclidean spaces.
    The objective of this thesis is to create a small video game incorporating both of the aforementioned concepts.

    \noindent \textbf{Keywords:} keyword1, keyword2, ...
\end{abstract}
}