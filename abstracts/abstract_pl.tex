{\selectlanguage{polish} \fontsize{12}{14}\selectfont
\begin{abstract}

    \begin{center}
        \tytul
    \end{center}
    Proceduralne generowanie świata jest techniką zdobywającą rosnącą popularność przy tworzeniu gier komputerowych.
    Zostało ono wykorzystane z powodzeniem m. in. w takich produkcjach jak \textit{Minecraft} czy \textit{No Man's Sky}.
    Z drugiej strony za przyczyną gier takich jak \textit{Hyperbolica}, szersze zainteresowanie zyskała kwestia osadzania gier komputerowych w przestrzeniach nieeuklidesowych.
    Niniejsza praca opisuje implementację gry komputerowej, która łączy oba te podejścia.

    W grze zastosowano, z pewnymi modyfikacjami, metodę przekształcania przestrzeni euklidesowej w przestrzeń hiperboliczną lub sferyczną, zaproponowaną przez László Szirmay-Kalosa i Milána Magdicsa w \cite{Szirmay-Kalos2022}.
    Wspomniane podejście pozwala na wykorzystanie znanych algorytmów, zaprojektowanych z myślą o geometrii euklidesowej w innych geometriach.
    Dodatkowo, kluczowe części logiki gry mogą być dzięki temu dzielone pomiędzy różnymi geometriami, co znacznie upraszcza implementację.
    Zastosowana metoda pozwala również na zmianę stopnia zakrzywienia sceny w czasie rzeczywistym w przestrzeni hiperbolicznej.

    Proceduralne generowanie terenu w grze opiera się na algorytmie maszerujących sześcianów (ang. \textit{marching cubes algorithm}) z polem skalarnym określonym za pomocą szumu Perlina.

    Gra zawiera również wiele klasycznych funkcjonalności, takich jak edycja terenu, animacje postaci, cykl dnia i nocy oraz wiele innych.
    W rezultacie gra oferuje unikalne doświadczenie eksploracji i interakcji z światami fundamentalnie różnymi od tego, w którym żyjemy. \\

    \noindent \textbf{Słowa kluczowe:} geometria nieeuklidesowa, geometria hiperboliczna, geometria sferyczna, proceduralne generowanie terenu, gry komputerowe, OpenGL, C\#
\end{abstract}
}