{\selectlanguage{polish} \fontsize{12}{14}\selectfont
\begin{abstract}

    \begin{center}
        \tytul
    \end{center}

    Proceduralne generowanie świata jest techniką zdobywającą rosnącą popularność przy tworzeniu gier komputerowych.
    Zostało ono wykorzystane z powodzeniem m. in. w takich produkcjach jak \textit{Minecraft} czy \textit{No Man's Sky}.
    Z drugiej strony za przyczyną gier takich jak \textit{Hyperbolica}, szersze zainteresowanie zyskała kwestia osadzania gier komputerowych w przestrzeniach nieeuklidesowych.
    Niniejsza praca opisuje implementację gry komputerowej, która łączy oba te podejścia.

    W grze zastosowano metodę przekształcania przestrzeni euklidesowej w przestrzeń hiperboliczną lub sferyczną, co pozwala na wykorzystanie istniejących algorytmów.
    Takie podejście upraszcza implementację logiki gry w różnych geometriach, ponieważ pozwala na współdzielenie dużej części kodu pomiędzy nimi.
    Pozwala również na zmianę stopnia zakrzywienia sceny w czasie rzeczywistym w przestrzeni hiperbolicznej, co jest najlepszym sposobem na ukazanie różnic pomiędzy geometriami.

    Gra zawiera również wiele klasycznych mechanik gier, takich jak edycja terenu, animacje postaci, cykl dnia i nocy oraz wiele innych.
    W rezultacie gra oferuje unikalne doświadczenie eksploracji i interakcji z światami fundamentalnie różnymi od tego, w którym żyjemy. \\

    \noindent \textbf{Słowa kluczowe:} Geometria Nieeuklidesowa, Geometria Hiperboliczna, Geometria Sferyczna, Generacja Proceduralna, Gry Wideo, OpenGL, C\#
\end{abstract}
}