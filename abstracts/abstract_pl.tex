{\selectlanguage{polish} \fontsize{12}{14}\selectfont
\begin{abstract}

    \begin{center}
        \tytul
    \end{center}

    Proceduralne generowanie świata jest techniką zdobywającą rosnącą popularność przy tworzeniu gier komputerowych.
    Zostało ono wykorzystane z powodzeniem m. in. w takich produkcjach jak \textit{Minecraft} czy \textit{No Man's Sky}.
    Z drugiej strony za przyczyną gier takich jak \textit{Hyperbolica}, szersze zainteresowanie zyskała kwestia osadzania gier komputerowych w przestrzeniach nieeuklidesowych.
    Za cel niniejszej pracy obrano stworzenie aplikacji graficznej łączącej obydwa wspomniane elementy.\\

    \todo{Translate from english version when that one is finished.}

    \noindent \textbf{Słowa kluczowe:} slowo1, slowo2, ...
\end{abstract}
}